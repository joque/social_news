\section{Social Networks} % (fold)
\label{sec:social_networks}

\textbf{Need to add a few related works}

\subsection{Characterization and Examples} % (fold)
\label{sub:characterization_and_examples}

A social network is a structure in which the nodes are individuals or other social entities (such as organizations), and the edges represent some type of social bond~\cite{Boyd-Ellison:07}. Social networks are built over an infrastructure that supports interactions such as face-to-face, phone-calls, or contents in  mail. However, the Internet made social networking much more effective. Indeed it mainly serves to reinforce existing social networks~\cite{Kleinberg-Backstrom-Huttenlocher-Xiangyang:06}. The Internet, with its support for both synchronous and asynchronous communication, audio, video plus ability to store and refer to past interactions  has made it easy to build  social networks comprising of individuals with diverse skills and experiences.

Within a social network, three types of structures have been identified by Kumar et al~\cite{Kumar-Novak-Tomkins:06}; (i)
isolated communities -- star structure of a network (ii) singletons -- do not participate in the network (iii) giant
component that represents a large group of individuals who are connected together through paths in the social network.
The study is based on \texttt{Flicker}\footnote{www.flicker.com} and \texttt{Yahoo!360} and indicates that the likelihood
that two isolated communities merge is low and a community grows by adding individuals at a time.

At the modeling level, the social network is modeled as a (timed) graph where edges are created at a given moment in
time. Basic models (such as~\cite{Kleinberg-Backstrom-Huttenlocher-Xiangyang:06}) assume that each node has certain
friends, acquaintances, or colleagues. Internal structures within a social network manifest as subgraphs that grow and
overlap in a complex fashion. Existing social networks such as \texttt{facebook}\footnote{http://facebook.com},
\texttt{LinkedIn}\footnote{http://LinkedIn.com} are implemented as a site that allows individuals to (i) construct a
public or semi public profile (ii) articulate on a list of connections (iii) view and transverse their list of
connections and those made by others within the system~\cite{Boyd-Ellison:07}.

Another special social network is  \texttt{InnoCentive}\footnote{http://www.innoCentive.com/} that comprises of approximately $140,000$ members~(\cite{howe:08}) whose social bond is based on interest to undertake competitive challenges. \texttt{InnoCentive} has  demonstrated that a community of individuals can solve   problems that elude some of greatest  specialized R\&D teams in organizations.  InnoCentive exhibits a flat structure where there is a shared bond between all the members. The desire to solve (or benefit by solving) challenges bonds the  individuals together.

Social Networks have generally been used for various purposes, including supporting cultural exchange, sports activity,
etc. However, in this paper we focus on social networks as a support structure or a platform for decision making and/or
problem solving around a common interest. Thus, the following section overviews \emph{crowd casting} and \emph{crowd
sourcing}.

% subsection characterization_and_examples (end)

\subsection{Social Networks as Problem Solvers} % (fold)
\label{sub:social_networks_as_problem_solvers}

Other than the ties, a group of nodes may engage in private communication to perform a specific task. Nodes participate
in a number of tasks whose completion depends on the power of the social network and strength of the social structure
associated with the node. Tasks initiated by popular nodes are likely to be completed fast and in general a group built
around a common interests will enthusiastically participate in tasks that fit within their interest. In both cases, the
social network provides context and a structure within which work takes place.

A task to a social network is usually broadcasted in the widest possible audience accessible by the source node in the
blind hope that someone, somewhere will come up with an answer-- a technique called \emph{Crowdcasting}~(\cite{howe:08}).
The social network serves as a network of potential solvers. Page et al~(\cite{Page:07}) point out that a ``randomly
selected collection of problem solvers outperforms a collection of the best problem solvers". This conclusion is based on
several experiments in which a group of highly intelligent agents was pitted against a group of averagely intelligent
agents. In all occasions the later won. The adaptability makes it pervasive and powerful.

Given a social network and a task, the likelihood that a solution will be found by a particular group of individuals
largely depends on the bond that brings these members together. Members enthusiastic about their interest form focused
communities~\cite{Kumar-Novak-Tomkins:06}. New social networks may be formed from scratch around a particular interest.
For instance, SETI@Home~\cite{Anderson:02} (Search for Extraterrestrial Intelligence) project launched by University of
California, Berkeley, relied on the public to share their computing power of over $3$ million individuals to interpret
images taken by different satellites. The objective of SETI is to explore, understand and explain the origin, nature and
prevalence of life in the universe. A Similar technique called Foldings@Home was launched by Stanford university
chemistry department to tap on thousands of individuals to simulate protein foldings.

From the above examples, it follows that \emph{crowdcasting} a problem to an existing social structure requires that new
dynamic communities be formed around the specific task. Such a community is initiated by the originator of the task and
extends through the social structure associated with the node. Individual nodes or group of nodes combine to solve the
task. Two techniques of detecting and accepting the result of task are evident in literature. The acceptable solution
could be on competitive basis or by majority. Under competitive basis, the best available solution is evaluated against
some criteria, while under majority, the most common solution is taken. In both cases, a solution from nodes is evaluated
purely on quality. In a competitive basis, individuals may be allowed to tweak other solutions to improve on the quality
of the solution.

In this paper, we use a particular type of \emph{crowd casting}, {\tt Social News}. Before we discuss the application and
its specifications, we overview the formalisms we use.

% subsection social_networks_as_problem_solvers (end)

% section social_networks (end)

