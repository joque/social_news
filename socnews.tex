%% ------------------------------------------------------------ 
%% File: socnews.tex
%% Author: Jose Ghislain Quenum & Benjamin Kanagwa
%% Created: 10 2009
%% Updated: 11 2009
%% ------------------------------------------------------------

\documentclass[acmjacm]{acmtrans2m}

\usepackage{alltt}
\usepackage{amssymb}
\usepackage{amsmath}
\usepackage{amsfonts}
\usepackage{array}
\usepackage{booktabs}
\usepackage{boxedminipage}
\usepackage{fancybox}
\usepackage{fancyvrb}
\usepackage{float}
\usepackage{graphicx}
\usepackage{hhline}
\usepackage[latin1]{inputenc}
\usepackage{latexsym}
\usepackage{lineno}
\usepackage{listings}
\usepackage{paralist}
\usepackage{relsize}
\usepackage{subfig}
\usepackage{tabularx}
\usepackage{theorem}
\usepackage{url}
\usepackage{xcolor}

\graphicspath{{images/}}
\sloppy

\newtheorem{proposition}{Proposition}[subsection]

\markboth{J. G. QUENUM and B. KANAGWA}{``Understand! Do Not just Guess!''}

% \markboth{J. G. QUENUM and B. KANAGWA}{``A Process View of Collaboration in Social Networks''}

% \title{``Understand! Do Not just Guess!''\\ A Process View of Collaboration in Social Networks}

\title{A Process View of Collaboration in Social Networks}

\author{ Jos\'{e}~Ghislain~QUENUM\\Makerere University \and Benjamin KANAGWA\\Makerere University}

\begin{abstract}

This paper proposes an algebraic approach to specify and analyze collaboration in a social network. We represent members
of the social network and their behaviors as processes. Using a social networking application, {\tt Social News}, we
illustrate how to specify and analyze the behaviors of members of a social network using both $\pi-calculus$ and
\emph{P/T net}, a well known variant of \emph{Petri Net}. In combining both formalisms we wish to offer both the
expressivity of term-rewriting formalisms (here, $\pi-calculus$) and the clarity of automata-based ones (here, \emph{P/T
net}). We used $\pi-calculus$ for intra-groups activities (group formation and group level decision making), while
\emph{P/T net} depicts the inter-group activities. In our specifications we took a special care of secrecy and
confidentiality.

\end{abstract}

\category{C.2.4}{Computer Communication Networks}{Distributed Systems} 
\category{F.1.2}{Computation by Abstract Devices}{Modes of Computation} 
\terms{Specifications, Theory, Verification} 
\keywords{Social network, crowd casting, complex adaptive systems, crowd sourcing, formal specifications, $\pi-calculus$, Petri nets}

\begin{document}
	
	\begin{bottomstuff}
	Authors' address: J. G. QUENUM \& B. KANAGWA, 
	Makerere University,
	Plot 56 Makerere University Road, Kampala, Uganda
	P.O. Box 7062.\newline
	\{joque,bkanagwa\}@cit.mak.ac.ug
	\end{bottomstuff}
	
\maketitle

-----BEGIN PGP MESSAGE-----
Version: GnuPG v1.4.10 (Darwin)

hQIMA77xv53FwjH7AQ//aSmFUZrNxtZsYMZDdZkeXgqqxT/fC8vU7apGgBHJE2Vz
yGsrN3bzT1xcjSNtr1tectjaUR6+hZPhYh2AiIZO5MnWx1rycrz/KzOGJ7M7J4wN
gQBFf8cGtOdvjb1oP5jYPfYPMJmXfx31S5vSm/lq8oXR83UBPggu87pWZEx6cy6J
KdNwfa+YgP2qjsOv6k2gGMn6rr0RXccjLA857VCImN2Lqa3hiBR8Yg6QS2fML3mK
oyWe+pGvnMJUT9ipqO1ncY3ytLE734yQZHOej+OTVfiOkQpdqrmyXPm6o3Z9LdWj
V0/FLboCvlS1AIEr6QvGFsirkltdXOSSz8vNmecm4rHd1GLk8XkQImA4FpOM1uiX
O0USGYu7BgoUx0HE0DLSGUJmGRFRj91OSLhqTTJH25Av20aLJH/e2MqKfC1MEsOz
f0CiJvkUAQ3O8UnCIAHyLw5AEEUpDrVmei3PquE+VImTh9KustZxGn+slGdpqXTw
nsxEOVC+veMVM80PqM6lcqDA7/jM9R7jhvqzacgBU711B3yppge/xxv4KD0Kzfrb
F5FY8lP93f/+go7DiaPdoeOHQuzo7l0OUvVfBa61UNjkFMaJIZ/MIXH5nxehnjVT
J8tfQoAElyCAaGjSLbVd3V6To1EMUGzHNM782sLFGGDEFcdPsspRYwcTMxLfk3XS
7AE0Wgh2ip0/jvktFssLOCdw07xfDwxmKZAl8AwSsyqgB4wDjzethX3hkaNJvISa
J/+c2QH2nSXRMUxiNes7+omD+AhXvRa9bkvaCSN5nFN+APXNuzvvfvKDXXXcPTTC
3Zl5Bc+Cd6dNA9DzcDaZOWUxY42JqwQ7mtWfmX4YM7B/PTl4pZVWCLwvh/g96Nh2
ZxmjEE1V8ahD/uyAdlBUcxBtL/WkcZqX1nHCWYV5oN6j6VaoLi9f64RMrDxNR4rF
RbB5FNp82sX+8qtYD0SPQisXiEfRwuU7eA1NxC85Lr+XmC1iH7VHeR8zN2Ed5Os9
pMPWiHgrVzpudWrVc7da0/ha1czneKUWD5+rRNaO4hdCx2UoCM/FbPfm0rf7LN5Z
65RphM8XpzAEE97TrOzFl+44bXTC/PgtZxXOdlpSbVyyUHNf1871LLAdNeY7YOlW
sfvQ0/4FnnyWLru/0ABAMXAOw2qqU5kdwhLShWS6ktEeRf8ekt8qTkmL30a1y+a0
VtfBoMzqXxF8iL6t3wcqso0a6veELFH+2sdlP5h+v2boYqXLSqYQNx7oaCahag/K
iaI28WasBniEto/YOrhO2kR019UieSZ5beb2gLm2I1EBfWzLG2PXXOlb/+ru1xVA
al4FNZU9WM3QMIbl2XFp7YwyUsv9ToL2LJuUTozBgh2/3uUOiaBkKsr6QshLsB6R
EhPgYiHF42BhX3yXHgqBX+O+ZRgH5EoUdFwUgAomRbjRff7dKpU/qIxMxwnRtEg+
s3pK0uJFKCaUfrHXT99om/ZOcbLkcxj931iOFdxrbyqE+KuptM4pOS51NfhzvcZG
nc4c/2HRCRYGVDAAvCGSR11GDkv3QbNYcRd17OyiMZe5AP8GX27y3xbOT7ADwq8m
iFE9SVRPCALDNUR7/yxtqXgH2Umnk8AACtRn3fiZiq7gjbUTv2PZtRD0n6kJAJ2c
JFlevO+kgl6uZRFiEzmo5VS8+w3OpZQqjCE0KnpXxhej2YGcEy0jale0HFK+vdod
Af9g53yOL2hKXWJ9R/DXj/sdqAzl/hEafHIZ0JhVKLGNO/oCubtAzfPngQog9Mgu
HQphMguisxwlUYRV43bsuu7ihI4x2Kv3MFOovaH/7MbmHHRLTUyylnylGR/VwVhF
FSYpXAW3LchorTvGqzkzjm3UPuQHTPlke9zWOGXdPjqjj6YOhO7cU1zLGenoSRp+
Bu4jYZN/Ax92byZVPg0c+7STztpvYmr9Ci0LawtMs2kWQKOeUmrpWXGsY6W8HTSq
kbPmVj0Kxjpi1aOmO3GecgE/96erZ+VBkgRE+0LH5D5t4nTOGmPyyjQXyQR7Tp/2
MNJJ8iP44Ca4vgTWRKAz8cc05J3uKtdCfZ3sjH+uBNhashS1BhE0d0FQq0KK72HF
2TBVs/KdciO58CmeZjz7zKbjTXapohP2A06ldFkoJt5Q/m04vJ58Jb5hbEV7r1eA
wOpwwGBKoh2PEMENz4lMnSELeBhIU8kLgDplyHOGIqZTG3Bd7coFNKHA9Y5UXiXX
ephnotik2L1fVujfkGQH+IhQVEPYPJSpgmEQivJWIG8TmTnW/beJLd4IHxKbztyI
1vAsUWuBJHCFUEyajy9W1dY0na4jYLZNyK73j8oeRaLZkYj6ybw+SnHRPJ5iHipB
D2c27q+h1NjHEAdRN9UNiXmulnOlMm1q4dahuJ+JKwPBnT40G8hOZttwF2N+6aTv
0Qy/hJ9Fgpu4zl0Zvy6kdcz7lCLHh5moTXOW6bQVu2HpTxKYZ1rqJIgFqHItb1k8
Z+WZZ4MwGWAwM3LcnY4z5ZQcQcBUfXTTRBXvVhjrHusa9L9rqJAcL4LvpF3BHMXV
hbbd+MZkscWAG7PorMVXccwNwDkbPD1XWZhJCAuUMvqT/B4S74oOCTFGuKp1IRVM
kAnkbHKzOMvZOZXEo/CU7l4bZSFUbwBVZw7xeIKKKCGCmTE8xM0ZzUgW/HcSb1cm
ug2aShP3L9M4V98GhsfZ59JOspLpVT8nX/9KxFW8ZOgw0nbkRBfketdIB30I2Zu5
XEGp2G59YJTslbbmQvrqu+gPJ6rMIXv47XMKPlSD+ZR2hQtQciQRNa2gFhpiCYr5
1AdYDY9aYxNEdX40yban1JNA/rFIy0lP8xgghzpd3BHb9msAZUvOF/XJcgMwaNXm
8F3srzvNYnoe9RK2ZzPvVezV3nNFXDWJ0Mz2BElZsfGUNcaGac+Mx4pmOF+GbFNi
xIONOERJJbeUlYwUhaX4vt3kcq8SC+CBEMT8s6aGH1aMEdHOoAChFKMHDivdPe7C
nNNReqxOHI0aCtx/+DYJgMBWLEgK1PEO4FkY6NjdvwDp2qeIla0hemt2BFB3HzCQ
Z3sxonHx+QTuGHUYzZa0VzI6kouoQT8h1w1z/Du7O/+hw7QvUbfLh6vafHILrG+w
3zs0sEWcgav0hzeDdnITivCt/8QIbe+mU8InXjqDky6u89kHBDZKC0sJ+BSfIxw2
O0dSqH7rzlS9bWRLciA5eJP8q0GvJpuwrKbpY92zCbzko0K+0JKmDuAp2K5LC4hm
4jp8nk1jpNe27YFlWYftsR0NEcYH1u2cPn1tOWFMndjDg5UQo0opdemHtlKJjF44
QwDh3C92a1qdEvTfbggcCJXXzwQLQMEIHEmJKkfYIDx291IOyETEIH4OrGBSZFKg
82/rx1GM0NKKNHySbS/8D6cFwy0O24UHrr441FV9rX9pMhFTcaXVeR850m70fqKb
NvzL6HxopIy00DG71rdH3GlYe74yvr3q89VB3O6ugJwVJWdsNiU5+fqcSVRroxW1
2+w7r/e/Xfd8jl76lro+Bha0+WLOCmnuT69KvIU8B2013rIJ9R2ivengpRPwZfMJ
8fnPH6+DKFXb45s3VvFpzuCubG969YUnDC07WAqLjaYzrBgGRQIqvs/NlORP++zY
/ssBjXlaS/ctViwsA1DylzXnzZx+o+ie9EnSs79Gnsqfnw/C2yFAzUc3o1Xov+ON
CacB8evsxI/hDAt8Mrz2uIcw7/HjLdts+3k5sn5yImbimAMS2RP+nZvXgKPyb1pR
jnGdgnz4XCJ2hFSFrQ/ibfEKPzI4ivnr17RqyDUr+WzT95mA+OpLWX0IH8p1ttzv
pHWMcvCth0WSxbkzrKSHcWubNIgDyifkIva/eM9kZJIZzCZTOg/U0za4YEFa+xj1
1Xnu4EUh3MUrfOi5urOm5P5i+jjXxfvXG0TPDJvWzvP1buG/VTE6K5TjuroFWQDX
VkKVvIBgMf8TbmIo9245n+1HjeUyZXh82tDiKH9YfmLm4+lXJS/6PcodsAPbqe8w
gOeA3F6zqEVcTedGtGEnFwM5mAjXIzbZZzeKtJNX/CuhR6EjRvKGrhAyUsa5fZVn
0K5XvJ4UZ8+0OoBaNP6z1L/Zdixk0nzrrtReiNK+JlxeKC5hq3qFmBl0SBvBwDyj
dvtkJpx6ie9tN4KSGvIhhjHIj7aKlY344WsKua0I/4BsbpubwlnkzBq0url72rbc
U9GaXsvdQifs8fs+9zj+15Q7QR9LYY5XIkcui0bcQemW7SDJFWDesG6Ui2uZIpCg
VpLEM37QLgPoo2uUnvY4ci15kss/0KLn9fanrPLNqwyD7mR8Mvo0kuJOjfLfmvnT
KlqtJ5aDmjR5fho0L4ZUoksmFduEbnZP6AoPnofuon3r2E4KtibK0dgBlQazqTKY
pQkhVSP3X3Xr0DnT3TDc76+wdknEoBdCF8kV7vxpAQhgDf2nMDL9C/TnyI6WzZxa
Ef6SGENE6TDbsMNpxKf5eXN7WrB4dR3czPzp0eICPXfg3M6+6DNQdfP6V/bA8evg
XAtR/TXnnFDqfOAb7uDEE2sXx9N04bqVCCCN4BF5NUEze0fbRWFSFP7inEEvf3M+
mX1qa9SOm37uhXMqfi0aO5OT97i61J3wYgnINgZMuVC8ZLKbY1sldeR2cTT3/8CY
B7t6QKvLS2nOMJAcYCZ5PXwF576bJ0z7kT9VMjOY0sPGqnLWnMctnlBg5/E488d8
neb5NzeOPVMF341OhPoQiwhBmcc+Nz+lRCwoeWZHGFOC3Cn+IIUqJud3m7X71T8u
MaKHjkoY8zvtEymFoGv2DRMiberjuJPlyucsGxMgevZ3kfsK1t2fr2j3fVZANuNZ
Cu/1Wz1DV/QxTkNISAwSwwfqNQsQYfrFhllU/2BvLqAQwzHfYt9oDPsBs/kWPmhH
5MPWkR6fd6klfA3uvr7Twg+CvSXa1G41OFgPKMjROs1qgyGWhXlJQEMFomyneBgp
m0lAaUxPHxWg5IVs+HUTKdwKFF3EqK9EbcdEUXNizGsr7HiZ61d9bFR/lrT8ZwmN
LE/N1hqYT3wB3OwoUt1tTvkTNUG6MiwHdWyWUZEPeJmll+FwvroHzWOlGKT61/L3
7k/4qcnhjK3rKbn3AezbSOlZhZ/OXoizOrN1z6AGFp0a05Hbhc7KqNzmbEVsCY6f
mKmXTYdL/siyGWc74FbESnkBS1yIUj6RMtfCmjfsk9B1W3/YpJIajMJJXndTXZ0P
vT5AdifrfALeMpV2+hkmKvg8aIRerTrtSVJWP5LM+Q5YFt57U+fqIBsti88J0Snn
8XISVfw7OvvWID/WXe5F9OIA/nsWUqJd+8mxeu14sDjhMCDiI7tkPonMIipUGgw4
jE9jsoKWgOawW0OXEzRcr9cgkUdvpU7SzzUElupInopr8osC4Ad4YY/TWU35aDFm
4BGnARrSbqnJgmho+8DSe9pHN1FhRq7SmdgCXw15jG7EqOadfPk2CgormyxRq4hQ
TY/WeaIAnMOgtoTITE5GIrm28wJQHkbh9PMQFrtmAq1VGScb0iIaNLrPDac96zpe
Eobf/Zo5ksgnRD4LBCfrnrZU5NPLELPaa9SzXidJLYzMESe06T0TrbJn0d4e6DF3
hzBxhR5rxUkwMu6TYzNmv7hqXaFVgNXUPdO9EcDtyktIkO9TS1Co/Xf9gr1TaYb/
zlFC4yMOkejdC9wSLOqxZXfvupKsFK9yDoHtl2RekYSl/4C16WXmx+u8GLZWhLPd
b5JbwDFIXwPqdVfHF/ruLR+RdfcmB6FJWzsAqSQKV2XgRzDOjtt2KiLrmpSbfLWo
V+KmESFTT+QFqup86+nkU7geh2bQind2YOsRALgyOOJHjgxsosu5z/FQnGUStAx/
shWpyxJrlZg9xQmzwRoNrJ5RBYg7pN3V9JaNCehHtvFOLEW3S7vzmdpi91yRlaxc
OELZDJM069W5MqlZ/Q+Q6E1HqsxMN5Jpj8B1pBXh2izgJjgMNYdpuVjzBALBlXKt
cPumbREiSgX2h8ElzfxfyvLtT4RzbW8aJLCzHJZg4BymUTCzH5HHpWvkJ9rxmJN/
YeZH2FNshzRnXn9MIX6+lEGaLS4FqH+hI8TA7YuwJc/u2Od61RnCp2fneTR+B8OJ
xtjrPETmcEPkduyZadINb9Xrc3DGbl1ZXW7kGkIjDyJ1Fps1uwd95QeoQzdwps0B
X3LitRmYHw6D9bsMSh4yjjf7hQcurZZOeevqjuXX8zf2olLkgX6blKtcJFxN7B2k
lw4pegdnmoGzyI+cpamD/uz4ZCfL6gWG681+oNBtBk5m+/0BC2edjcOEJ0Mt1s1s
iznvy5E9/m54EqZt3hNk0aXCkJAP5rXKtx5DpnQ9+UNMpZeskKjY65HRsz9G0XM8
x3KRWhGBfGoRndzP/5PB77jFwAXRjCJWOIHAHOd0Zdps2YQPmFXXPBZ3GmOzZuS+
/z7OWn1q+MonkoYP+fy0W9189fKoNbcRaXOb9y4w2MhMyECRiMttW2wT9agWrCcG
KfM4sDtakNKnEyuCVC/pnlkXxdGfCYomOH/+Kc3sQ6s3vcjjUP7wvjUTRtH5Qj8R
GSUK+gVkBeK4TVraUv0d2qIRyLZB12kmjC0Wy530Izjm6rPW6GCLvZ7slqxW2zk4
AFULYDpyOQBK2ko16HqEXWTyIrg4mZKbvinBqZ3MuEqS1mowMzaMm2kJjTG5dYJW
DGaY0swIZZLq0wC2KQ7aeRdMbOFEw4SC0YyPZENvebnQGpmbqlB3QontTegSfOgL
EnxQtMhlpfxIw9Cci9fnkPHPicy3rgqmGal90PuWttFFhizAvdhlf8iQ4sKVeo8l
xnomI8RW9cyUyd2mXN7AhLvBVgzj6LwfkOlfEST7U6ij+IK+djczDCZQ0EZAiIkm
ce8ZfSVNlvxY8ZXUuuQe45OjFNk8SNELB4BJ7+U95bm3EBxfmN/JIGbgceq8Qn8s
gS/cz6YjT7h+oSdrTh4pYdqymB7jVGHwj0Xnw5yJlMdUPqqXXkZ3XpRcJYdW9K9J
z9kq2vIbU62n3X90g6nuPes9uTVUylIbDcenBjhYIBHsYjXeDzhJUhFLsfNErw45
M+OoLL8L/4cWBoVd+MLlAbUldFwmMoxlc8985Gz7v8pkK6pO/i3J3wCHZFFJ3Ev+
cGBd7fRESwb+ysAq7XKDnxhbMFBNAI9fcfqOQbhSOfIu2maRehpr6417UjnIeLec
rS/t3+uVvSytrtBHqKnXYnPyj2bfJDA+iI3zpWCRDIi++yGDqFKExIlDISet69A9
c4sCumUsurfYiNx0l6pKa73+BFvDZKkVqcmhbt8jwxb+vTvmrLeIC6IqeCcmKsZh
ADrm/+vK7IFCC7YU0TwcdClNDZoQbbcnn1yMvmPSUubGh08+fClmfM9sFPt/bQXp
ma4XJUs1FB9AcTZBWRCJku8BnQ1JnNME4srMn9fGxJ3KBxZnKTg47hbtxR0XRhxq
wDsGBkXax8P51wHjyhZdm3bqmo92R7XlrRu6yxZeUmZAq9/KRl2O9LjXuxZMHvC/
27u0QSlA0AWDXZ6udby1qTNHgILhvpIJpf3za4vyl38FMWFf9UgNg7OW8Nwp3rBU
SmPciXFF1bXtneq97HuO+6vV2/A0iUrVoQebiQTSziwY/J5CgT+a6qs1kDhNWvDC
YMtCVBoXXHStCitB7sNaBmHLQf2dhTY7B65YjSlG1050SbE6lwu94r33hg4cYj16
EXwCbfnbGLl7BwYsH985NlFmOJtw88xwQFAG2Pqbftm/M7hhnKk56uKsp2usH5oz
SzgYVn+Qv7Onb7iLCTfU7U8YHDVcNMof9HZmT7djEG6QqnZl1nOHe4KqciDmYmkz
fdd7lnC8ULWBkcof7HxsFurEBIIYCYELbbqkPgBCyox5GVfMHikqsHs9sX65mjKa
IxIVC6aNKej7HrpV8+aGIqXGSjfUPLKoWZxK3Eza3mkKYmpJIwAqo32y50x+5XKu
t2G9heq9kXxlatzKmdveunFyCqKHbTeMUkPfhLdlWR3XE3KnfknOhthBl3p+iYvN
/u509UHNpKUookVHNZPqqDIwRg2jpXE8w5qkUgTCftE3qnEYf87DMfDzKG/N326d
Bqubk6X6IJLsH900JOAVz2K13MMBH9XcVrxDsKLo2y4JTX4hV9oRABjNvPxDvfLA
Wvd36vFxZwgY0lzIaA5uINt/BtZCl6GaPCxWMULqGvcq8fC6fQUVuTuLNwMP/EuU
b/mCDhAxwH1FFj9/a8FUnR8idqNLtnFtSFzVphAjgCVzBx4KK5JhHRfZrdgQ+pCg
xh64oFfGWuBPUslhiDEs8Lka1ejomjCIBQa1QhbRRNb7o0cDQZgeptrhoWdgOs1F
WeThhb+uBsgw89lgVkvgHeGQlLLyH6vA7vUDoQSuPHi+h1QXLWUl16WpZWOfWwow
qBpIX83U/xPY3sXAFBizgDOdClJsTPjaw/FEw6K+d0n6KkROkyiRtotUuIRDPFSK
AxTX4+mXdKwiVoH6hUqWMQrUAS/2rkzNYw+3IQzrfyjcGnKk1/EphPOTsZX5sRtO
28Y2LCe1xalctxtyIZ4XM/lewhN2HAv2wgTW3Jos+FhcH473M4km5lq9nyprp053
z7IYPhrsPnZYwsVVloB7KrxBbF0OcX0QG9ORnMRQtf+gVhiVJjR6sr9zIst3rqe+
0s8B4xXZKhHOWdDZ2dWoaQPXIO1R0QeaKVaaN3LJwVdV4bRveNIGUfNvv8wCJUux
13823fxcD7OkjeeX+7xZD8CbK90sjvpehMqRuGHtaWTauyFxOah/Exi1Kv2Z46OG
H8/ynba7cJT9lWPN3+LgVI935YRXYgl4pvBUn0DB5qiFc/78T12ClFvBb+UlWNZu
ozPpKPLzUOdONq61bKR8ErSvmlpBHfKDMtsKeJ1mrtHr373mGpzF8FJpmyIwe9BZ
LujajlEfZt0P4DXtPcvN4zRSWyjyYHBvoIz5FK6W949xaOI0CqBpaphL5GNvA/ex
+LxgQSYwjzMncPAFO3F8QQKKv4IfeUBIdozOWGbG+CLVjJ8YVnLVnC6Zp+tzidWJ
bjVcvswVOrqZTtaR2ymlyRuTSIQxZJFHXpO2SYGWtQYUITfZcHbKHBJm3Fp4+F95
KJcj+JI/jN8TQCiZi+8QXAvT1FxgL89SP1suX7CXkNKuoYEamHmm+oaEbibx0KNw
RDC8W6LMZJ/3YtCmAHaFEWcf7EWpCNXC5B55/kGGYBuLmt6CGoyAyM0Pc9uyX6uX
VR/7YeN8nyWco8k+cZtxW6CI0ISJ0G7STo3hJipfto84wlYmnGrhwLctgtB4+YH9
Q1kkJnYCtmkzOrk/jhhaOQqC7ElMw1nzhufbMEnvTQHSLaJJelYrw6/RcqxsOUIy
DSYHdwOqHd1uuiBl1G8R3dWrb2QrlKgfYJ2Yz2Pw3kguLeiTq3D3Wrlk0+XqgLRq
BXXBouPs1EhhEOtZq1rdm8I6jbyyklnZ/SZ13AkqfjeE//smKLqeK92xc0mzcNJH
R3GjHXZYGlUEcN7aYeN5nLXgPl3BQHpVCP7lS82RDBrUOwzAb6J0M2vpMLb1la/m
F0kK4UbuWtVcGqP+EUtN1vLcV/+yG4U4/6GQ7GSMu5TtEle51rBujyohPMEHGcrb
O9couekDGjfDuKOeAOuifY/7aifmOiAvlkMK8S5ZI7OCU0bGenZ0z7zURiahDnEm
EWl9PoCLuVmGeQrmQVZJfAubUrH1+heW7+FAtcdpbixycBl3opg84/lMdyYbnN1q
JbTLAUrpB31FAsRY+j+/KAXaPREq+yaZaF4YIG//e36zzYYNUG5xjVQfQvz/eokO
mRjtK1dYHeHTj8PuFVNaRUE0VRH0Kvp3g5+kr4iYtd8vZ0CMqp4bOmo52+Dcm5fh
EbgPwL3XGsGQK3ZLvHgwhXIpYxb5Occh+909XAuHm9cmAcHzDxGubp8NdkRHuY71
1TnFww==
=U8Pp
-----END PGP MESSAGE-----

-----BEGIN PGP MESSAGE-----
Version: GnuPG v1.4.10 (Darwin)

hQIMA77xv53FwjH7AQ/+MrQHiOdKZQoTY7faeIX+jnYDdnYcrPt/DxzgH73XiMf2
pNPGSTIgDKthYWW1cPfX0PTRPxDMS0ML2ifGGm22DKZXUq89BrCGE1cQR3CjrUEK
w3RYqgnZXWqJ6BRZVkXV8NgAANhNrjlKHpQVtkqSh8te3XMn3rydLQLcfh03r/Jc
9Xw161tqNHc8PIJ/D/71/Yhj+pUQMEAMB4m7sQ/dcdHf4Cf6qwSbJbinuI4KOlP6
YCZ2GgObU5RBWyZ7We8zXccThxidep/X1BoapgXKueydAQkU2+jnVYT01iaeygeS
CWM1Rzzs/6kmWHwbLiA+Z4/sO5xNA/cUEiEUbWhuxwdfTVY6zbgNRBamcrZX3w3U
uTH/S6xDH0ak5vcquLCUWC9IsA1BXOjujbIrbPg36iM30jYc0fcgD6tHV/wxZ9sP
SCLgiBZD37aT1cqkTDauU+RL3kheFJmcXGpRFn7eduKJBaNUXTFVkfPybP/aXe8h
xtvzdUdYQLAaoz43xKQzVXRDGzo7JpelgIlAU5etGURrCt4mHV7NEFOFPs/hkz6c
jnE6j82tJ5cFMXoU5pfurEuy6apPamih2ZSzSHMBGGGEbc+s2S04Wp/skVK4cyQj
9jTP3v1DwErkyMnBdM6/mGTL2fY+oepXeoPhoKa+E80mGtGjhwE9QZY4cSCvhyrS
7QE9AFQ45UgeD7XZ7QZMlmZjk2xtIBs6IIZDFtTZUqI8C6HyfRjWMBqi83Ahq3UH
940O8jObgVZrXJGcwUYzsCsbHdyS5Suu/4wCATTR5u1G9XVMdVNHz/qHePyEgRHb
XveEVhFLdX51EaiHXkgc+Zhsg7QsoRxwcaW/9VyjW45Vs9V2LLTEi3Djidc18yNW
9fY9cENsN25Tc3GDnOnmw9CEzhQWvH7ZArt0ov8Pc+vfiT6QKUxCUuXUp8GxKzat
MWSfw6366tk1SFwqVx6zIB/+OnOEVGAvNnJWoCHW+R7pjhnzCVylujO/fyfmg2JK
7jzjxHEOyIBheMaHxmJWy0/Yftr2vmu6DUyQFV1V01HM1LXd92W6UvAMqJ/DuSsf
Gg041G4iPIRJiA3zsSwGLvjldWgSZ/V9ZULKnnB/K0HUyt2CIaj69scon5wExbvg
y50knLZJ7uT6fcxFGq4zJAOOicu8QWLP33aqXjlhl8uA8hcMG+CQibEO8XyHlyF/
bJRjOA+Pdx1o0KIC7yImaOFBsLgVTwN05piqLG+WndON9Q5VVUAe70IrmtN4oF5r
QPFzoG2/wCrqNsr5k4GrllzF9UW6CvUdhfwOgPNZM1xGuAnxRvQuRMxlgbIEqSSt
hhBdK1wwv+c+iUegw9TwRuXfhHNCkcNHHoezUe9q882H9+Q1dCmhGjKGDBri8Jcw
tLR9CsgpzTArY5E2V6WdeWL6+imJxCLNnFTCi/KMtltECIZLM1LXozv0M8FkmtbM
FrHmYEKWYi1ZR6HgUIqun/OiJAfHJkjG2ablhUh/kXr3GcDHvcGZqNeLU1FOJOg0
dWWHLmEPqIrNDHeTGY4nPHRzMR1bg82rARaMPKlROn6T00YbCBiHjHCtxHBhjaAi
sOGNSs4/I/Ibp3J5BYuXFw5aQsVYEl8hekB9xzY+QvU8vZeOB2JsySgynmnvWxrL
tQCWZ4SIpjwz8jmpjqlMb4UOE94VRZX7s+sgP7W0tqjbCscBzmfa/I+Yot3SNzBB
Pdo2tvHxkbvSjj3w8gG1eiL4hk3d4DFwT9c2Z+xLVKk91K3rYqVWz5CvIOkmA3G1
dgAbJKifX5XKd++MtYdrRqpuLtgswjNoIPr9yrohHws+bbjOkkYwwsk1LTIhPZ/1
xFEGB5+AmMf6C+GA55gAqIrVOVcSatFNGv8VkddByGpKdEvQtRha06dDpQNgNGLo
AQ5OnjXqENx0XYhZkW5fogIr7WJZXPZXp5u1gcxpAeA4XGzVC0tuZnDwoJPAvKRS
Ztg93EqnMH7vW2HCy1YV1DyCjAChS+q+C1a/OVnvPHrEIs2stB72aOHRpRD1xAMc
CWTqtGARtedDg60WbeEw+MspsDaTzOLd2VBQxqrldNWLMu7ReTaL1kRRzcp8eill
I6pGAncieQZuq5q+I13PYX1t2HeM2Z6HHGtp/GQC0ahUMqweIK2/p+d/ySiZgAL4
hSvU9Bx1/y+sqOfkH2XQB2uDysaKoVI1FrNcfySV2PX1QjbEjfH9OmzUon3kLw31
DE1OqXxIFZ4PEI799+NhEuMrXFVGwQQp08DOSxX96PMgJWWGajv/9Ym+98S7rQri
CfMQzoXN1PK4CxvC7jczyZOa2uFYA5eWBEK3461xxIQCPsYx134LJE+vTKCOyZFc
5z7ignYcwwa9gNqAaIGUBqfDA2JmvIltaI4ELIRajwaY1qOiPdaFsmbw8eY34czQ
PsCx5E6LSASmYlwwuDYW6QkGEdHq9PEP6pYAVN3cBWOEqEq5/E/JWBr7W5T2Qbcu
3PnLaNQDy6taBQqGL3Ecm/2EuQG3BPURcJ7d+ohIYsZuB/8H0qIfeySBxibNJHdA
iq3G9ZpEnvHT13Moikssde2BVHRhNvgmIkzYedqQZpvSLDe6JhHgtGnonPDrO7kT
iI69jaSO4E95f+LHAZshGddcy7/vYeQSrXVCAjE47mxBLb3wVettm1ZBFR5OyIWv
oPPNDDxqJtknkaE1GYmXtB5Taz/lwB/k7quy9DD7w4x8s9Dag41ZKhLYQrEPI8MB
Ht/YJHEYkaHjMX9rqwQa3TguC5NbDOr465NdUGxLQT8Nhs8dUuDl0X/f/LOghvpg
mLo5QbUrfFjEB/TcTu8fwwobn5N9L0HtCi2NUAtNpFPx/KZDYkc6ArjNPOuMkumP
FkMPusg7XnjHTERUnmUWnIAQkBzQ52R4r1P+9zv4HGsuRzg25O/KiI6DpRGV0lUz
lKemrpD3aCmDzNgHJ30oHq2EjJW22SJX5tjB1eqA0uiBzcSUOALdK5QlU+8m0RQE
kk/v2Lphbd5qdlJhj2fqno/nTnahxPxfgR0IQCk60BxI+EK//OyiediePBnqwD7j
d7vAa4lUT7DXGXHG1svnCJzoE6jyq7qVb+YrLkqQPl9Gv34dc8vRHkWgi7SEb2w1
5SM0xUD5o/quNjjUJMZWmmuqK3TOhztepKv7Zm+URboxeBKprNvBZTsY7Cgs84N1
4trqOaSbj6Zxyn+l8YUVpPev3Ggi+3aFliwWS89PbWNySKXMA/KpzuFbJrLEBgbI
Mr3RMEj18GG/dcTr4s/5tT2r7ghcvXE3HH5tF3ahWL/4bYEiIsRr6N5RU/MLySOL
+n40mXPFuMY80nUgRG/Qvlj3EgWdXzNwSwQsAOzCR6irEAUZyytEjClt0/Yy9FYx
xZZMt15DP7BRNl4NDxPyTwL8hJwrHxFwUepHnFRBRQQD+dks3/fVYevBJ7+PqFkK
5Q+yj8TOkK3IBPO3lN3wOaKd8Sb9KO1nGvX3roNa/rh1/BjU7ciSkDqO0EJsNVOh
ra9sJ/08BO49gTMqLONmqPcL5mRxZwv/JUp69Mirz4NDn5Xd8rc8gt8+F5mkpA7h
VYUGQ/F2a44WT5k3c5AvmJNVYl0K8iCgKQgcrZdPjWUyBAGXytS+w2t3RFLHViHc
KntgPXyDQCKggng9EPN1TPSCXR29R/sfQKzrELpNDRBXY1C7Ju/FVc5cUTpwbdnN
EpE2vCYuvIpQhqo2vaIBe2YYoUiiFQByqcUEAN3+1Z0+waFMruxTJ/ezL85SE+0o
zta8/1MtIW5D7YNSlH4VgiFmOYZ7RbjEypZjiZhoxcWMEsCD7YvMtTlKl8PbCCAD
YJu4pNsnfzDYCYEjke6rUPKfyFJmdF3LXoM5OmnXez5qDN2jH4wAbNxTPfzeY/M0
MZyZC2/HzrVk4voHx/qztVUJ/f7NFo2CEyaC32CE4AzdMAMyCgdaRapmKN1h7yTq
BO69doTBzlNXsP9RafjOcNH5nN+GUSryRkxFXnE+hbpBNBOsYs2ToMcKk18Np60D
lovXXoLDUlGh/cKOS4k35pKFQQYibn69zqDz/BSPjLHu2MGGJKgkWyojkGPLDcz0
BN9Fz+Kb2AWQTCUNbG1tR6CzCEpjjGq2RQuXAK3NvYWh9eLUqJcOPB+vhlB60k8s
Y2uMzBlHfoE3aIZQpuqfLskvnk48EevZXHeAvUZoq4TISX/ys6c/PT09xWmHyUsR
uN1bLI1EyxchnK9VzQdMHNqmIlpQ2kJm4d/fFMsGrCFr9p1Lr/OuINHR0hDj+tTy
HuEwbx3s37S93A91fpqBNw7b3PxLyh3uMASRPly0IvianEDX5e4xtWEP4fKNYzkJ
F0Y0sLh5hDhv8a5KKHMDMA/W/dcSTaTOrThzQ7FUbVeMTiOyAxrd7JE3OZzUEgW5
y0yG3lB0h3TJGIrMvWsXf/u7//88PcJtzz9mMJ34g0GIK5w6h8k/JGt5I+v4OEbD
Wj4KTdq/HqNKf/UfoFXcg+y6xNPhpSlQn6M8VriiukRJcsK/5VQt2QBn5SmDrHvI
wm9kq6yOKu0NdKv6LYsa6q9piXsAV27rwGS9WXHfRammXZInZEIr5g0Qq5jQ+6WR
44XBs+W0QPKTSaBYQKJwSJAvmTv/BtwAnpL6VuqmnV/OXmMWom2kWfSpfbRGQCbY
TWVSxBhrtvXaATnoA2H05UXU+MATvTocyU+L6WuSP2hx65//4+rs561XufYbJK37
ZSxZ7djXMLKUI+gnEqk18fUKftiYFCAkUdHU8ceLuSSDiTXk1puRyJyVWsUKFPcP
7Zx0pkkyX7FjAeSjOvZTcIk21eagOemhyWqVdgBoPnyq3cyoFZyk3I6ZwSDNWHQV
bK/2y5Nz+/qxFtt1Rlz1KYqWHY2UgzKpHXqLn+ETZG9//d9xuHlfxWRe2eI3198w
qgkis3Vmv3o26sOS6yAX4f0HOqD1YKjGfoXTeseg0CuvHwTk0KC6Nub/hqpWYcLu
/D/2Q6Hs0odwC5TZrsIXK/gn2R87ZOSUekzv1b6yNQOt3BBP3f1AoBlwudHtaFJv
czv02Lqv6AYBkhJJ+mSKjTX0W/1BSyrkHZ0nwjvvhr5Ju2txZ/0Hnam0MFmDpfXb
kcAWYlVdP9B+BjGmJ7MvSHREhb0jXqUz+M3eTB8+SjRcOldXaCzjumLGPGM7F83N
YwN2VaVIgPIaTlsUxlpgTs0RqeaCyKLUQj5AY31cfG5Yn7FNNrVZ8vE5c4G3uN8d
zHogYVu1nlrZyl4Q6usbMjGAD32Adpr8mStqbk+WMyWHdfpujes01og5JGCmGZnb
3MAhuH7pbf2oJ2JNdmr2iXGK+keq97VCXqYnAZVbgKJw7THdeRRg+gpsYQb1qdzj
cFXE94Bv/dpsWNRjjM/hTWofelqVOrSvk5S6Rq/QjocaAGt5HfxX8sz3xEvy4FnG
659faqGlPtH6N4P0hNpwBQIMklbtbmR+Ko4RSo+5yzXCS1npOXnnus1Qhzqtsxsz
5xyfotLciGCrOGS3UUMUzrv+ocoOi1pExVflZhjcD7SpVYQoTahnPdp3en0UvbMg
uBDi1qUgd7PBZCThsgTCjmbpiZNPxkLLZT9njeFeOSfL3dYDi0LMvzMdfl0dB7Ro
JJRfTNNAU0dzyTzxM8PsCjGGAdLYWNuvfE/hj/UoAMrv5cuPe6OBcdbNVv110HwQ
8uO5iIQ2P6fMVhMWHiZAqIspmPj6rrXe0hkYcq1kwv5rWIrVBvtE8mO6oAKmCfnb
waKfs3YewfZWjwLgJe+2i52VGFyWcuEs9b900tXy7v18RaS1GdPwkRf+q6ehcbnG
bFc0n80zZnZyIMyO0zh7RpSD48GZlxN4G6lg4+o1p0tzwu0t2Qk8SmjDkFQY5DRu
3AiubaquwLYZ0P8sZKQUY6kah2q9FPBbbOA4YfqMschuQw838E0oGln+dhHOuBYy
7zd5g82Xmw4bUMCisYx0dBoAH8x0KTBX2OUgopHw/UmG4p2Mer1SF0bMH/BnATpP
oUDmE4x70DONUPKEG0BzUhObCgFblkTIGDm/ypQtMkaEMI/4UHKoovDJX6Ut1nLn
Z8n3uXKwv/NBACeXYtWP2M/Fa1R7/CA6KD5nbftKxnLsnBaU0PLenCthQRkHokh4
mM6bkhrgNioEDu2GHtgHtNK8UWNrSd3z2n35lMZNz6axTNBwlLNN8uoP+kecTru0
+3DV3NdtFbHs45goW6xzpJyEL9pcEouhLjukLFeH2W6GxYb8i2RDcYjLUqwvuCAq
PYq+0KmJetH9WpF9BE57r/cErfL7pr9H5GRNvoY0tLcSAgSj9lQdPJb4/gT5Wvtr
zj/UCvf/r81N2nDpnRy/mqDlP3Ye2s8ArcMLMYFTOKDLyTSIdjL184gSDGII4UAX
1u0Wd7B9cYEw+qU4lMjpkKvpVlKg0+03K+DPB5yk7S1/Vkih2EWjCZHYUXdNn+fS
0ckIZBR8MVBiOHhDhUO31x5/jnGV5XTIfZBYx6zeITkdeZ2m7RSEi0SuE3atuVzZ
Vhe46Qic9ZFO5s5yXMHGl13XdQvi0OVSnkuynTT0GJmFr2F741cnO75iCLZpGt2D
C0g1hC85YANNprentKmLTALAzXWMDPbGBuKsaUHoD+4no57iLEkCuIygJS1CTdAW
wCu5uJvw/rvPoaYXd5k9AHfORt5rqJjvtMQNLGMXhs8GKcKvi1wIBCHJ1wP0QUfY
uoW8sTSfHYuOz0e/rhtgb3mM7B9DksM/S1sWmb9VZ6WZb/lQ4lXcY9jcKM74lCMY
wREwlEywvmXxF8GLhKAp7oXsSkMoK+uVHDZNdh0gLc/zIr1wzKXRmhT6zu6MHlI2
CYPExopO8e40WTIZhVEVCulV19Z3GH6ua+TQ/0YoB8lxy/ZyS8N/jzArlym42jV9
U6PXrmEkFFEhseq7VTveqKBCoUnJ1L/rsLUXqaYjZLCHnes0MGEO7rvtp6fhc9wg
joI6Gmt7PPJwdClIKySX3v3P/w0vjnqXOFd6sMSnw6DtU/khGdmIM6vuDciv7A+S
JQSDAM8jCuPzRPfgEoBL2zkU2jG2aAnY+P+S46w43Gt479Ve9SI2xaSwiMP1LEem
w0Xr67pPAt3wolfnuUPV3ghVphzqzuGaEB0TO0BBxZftRsa9vxTqtQ7PLa6063pm
7ymTDHx/O7/d3i/7+JsDJE2JL/PpmQZD2RktLUJ4goe4DhMViVeedOL2EnDLXXK1
ZFe/rQpwT080RDYzymLBujL9zcV+/14c6EMuMcXav3b3+CIrK4Fjk3Kw2gPDeLtm
xinA0WDrfpaENpETY4cLxvxiy+xJgQ5Tit0OgeVeUdNMm3Mpnt264iynDC3S0Ny1
uYTTlWwGH5ltmmq2IBXYMHJP3JW1DKStcB6Hjb122bqitTQKqzEjNOggihUPR2lv
3zkR/1Hr3WPQqzep59QkWgLBkybN1m/lC3HZ/8+n+dH+a6whPfESZlOdg5F2pEnx
xx5iz+Hg91qQFJ9g9QFO2Lz2nEg/d52mQqgFa9tgDsGFqvLpK0Jis2dp6jBUY4x5
lymjmOeskggvECtuKWMfk7GwgUEvQjyV1EkFt3GKovPi/O+aRj9cFgWJoi209x7v
qvTgqq/Qq6Ejj7W5IoLzu4BpDbwHNAO5H/5FVP6aJuiV9zaOKqjJgBI31HwQ49Vx
3Jk5lSIXUmDvqN8cHl71hM4OLwKPhj4jJ8udP4Qm6XVcejEcjj2C2idQdLe35/MW
d7/OImVYjmz84Tvv6QGsdO3lsGpGg3ef8HGyJHCnzWl0AAB4aekSO2c7b29N9kDv
fC9oKIcyVl6kxSy+efGY4iP0j1hcXYHm6EL+Gwu8WSVyK8C5queeN/kiD9JQvr+s
Ca5jfF4MH/gJK7MeDgfRhjOVIquGDgaHXBvLjmMzw7HABIDJqmevCz5i2gfnFET4
097JSaZ+WGNzD0PJtzwFAOEbdTdSpoil7AvydpirLIJKfXEjizQCFTfNh5LICrQs
IBoIpkA1Rn0clQxIacfNSnE9ft5f/Tl1A+m2+DmS6j0KhAEILRyBurPZTUBq4SiT
qSWGrRcw9VLiGgCU+57EEljDgkTvF/vytmMPkBimC9MLKYIir4o75IFqCvrRwWng
uc/rVex2Tzzj6ayxda2KspaPkRATEUcU596J95mCeINVQryObuE9jrLhujkETWDN
KFAQuQzJCH4epULCNSRbemhBmRPiKVpflorqaBKd0PrCBN0IXJg7HIPIzXcijlZv
7RgXRb4B+8RmJJ8XlMd6ej//9jRW+XDWWxnAFFifFx++vPpfSvHEdFWi6Btozhvu
Wf3Vl8LfmXbvhIGb8kr5NTkKLmWIJcJQwF6l5vd8bG4O23uJ/hfv/SnhY+rKSIK7
8XzKO63A/1pleKwdgCaEOgCF4SjWGkFdCVJVeLe0aB1pHq02uOSNNHaVtL2gJNed
aX3Mte3w6pWl927k+qLLjTrnOpcHyAiWWYzjq6APH7iVYUH++prLdLN172V+MfuC
Jgwz8+JtAqdlhX9lKgg0r7zLsD8MCiDIQ4w1RrwykTRtD08soPIqXm6CivvGlc4t
R3LuoXstyQmeNmHN3K6qZxlnG/6fNJu3uH5bx3v0tfibpa03iiLnkJGtT/9U8+P0
gX5L/EGkCJCzzxiCWhJQkspZ4bEAEo5AxGh0yZiqcHncgEIHb7/5X3Tr8vz2DnmD
2m9Kzg+SWhKLvbsN54F4FDDS8Imnw0qpCW7TWMbMQJpue+DhYeSzkHgXdNgBwkYw
6/cI/mC+Zy/uuYZrBjcrgXkQOr97Jjv1eW1Pmi/2eaQj98ZDWjAQ56THQtwW5Lvc
VEt8CATSNMPMfUUSHS+m2EMksslbNK/apuIaeQ3d/QoPYNC4d0yEyEy81rk4Wm48
ti3kqPnst6Qk/kwXiJ7O2bqJoY5fZJcqUpqwHWyecBQRtSIL7ejd5FZYB0RT4pZG
SyOze4+G66NwlryPus9Z+gTAsWxxp+Rb8D4wmU8MOqTzR0Z3KfqYpl3R7xGilSJj
l4RzOYwCHqnKMjt32y9rFKZ3Jjhxie9Lh1T4/gHRmPL9ja8DpId+ZNVRG36FZKWP
vH32GE8/QcAK/l24t0YIY8Ds6V3ZvFEue3qdHIAthcjsvVs1FDRo5Sf0LvUeISEq
CpkngnXvoaj26Z1HcVBIgO47DQVXLUItMUVLn+KWcas7r/Gob37EISAHGI/+v86V
sx7foQuxft3tEUB4/rb2snmQclDFmFuCJi90YMx47xj2DheZUUXfw/xptEMaRVb4
b0p2vu+8Z2M9kwzq+cAr62iNjMYTRU1p5kdMZe8BRyhU1SxWr8xZ/j559tqUEn/P
b3hE966W19cknMO7+DXcuXWnPmQ1N4uylbSSsyX10Y8DLF2+XotrnaGevf2ZS7/L
hRQyJRfoEccLQYjuf68lW5S+PoDTXSd0xdok+p9R2+NGwUj3h1HDWSHW8iZPFxhR
zlSkQ5V7QcGtHsgj7JdzsL+4pOwVADNdM7dI0iqqnLAPzitGFr8/Zj9Np2dqhmOm
934UhfIxavGhEdlltUdSmf6MNUWd9VF4LxxEWFoITkOIB4Ri4ilPxp1kblsu+hDP
UO5SVbsAlTKhof0hRKiK2Qz0kSD/lzHDgou80273AEeVN+5QIVOU5CYayHGI9Ozz
+gImYhUMwD6ud0gXNmvvieZYjEXGvYF/qV9SkhFe/Q/NAJzhjjyNTkycrwVbdobY
LGJa7aFrXLIefdQZMk/sliF+dktAJFAiGWNgMmwlSa4gCWb1G/lmQp5SKHqiQGZb
ypq1IOcBX7b+7thVCPAl6aJsMsVX3wJIcmx28jdbfokH9xqY6JoHAAOGtFkvrqHB
aJlbJnxV4X6xTyUJUQ3YH31j8FuFCg4vpJ7qurpMwGZCakvMuT5Q1Fmqb9vDcxG6
amAWM9XIdDjjAmqNOhz56GFHqVMnOoLLd2QPlgXcSvefqzeSt0yRqB8C5SVwQCSY
kas05+MjFaOe7GK2CAzIHSRuQN/QT5YoGB2Fk+Z28ar6gVDUX4WSEddd2IYxVmvf
lz8Sbhl209o5z8ZSbYHc3cQDqPfHv9nlR44AW5cxZBQCgcp1UPmm5oTPqMesZk4E
5+PZMIVWsGIJa11r1DkphBuVlXhHDU4miTXARgd1RuZkRQ0pMT2feyNKtDbND+YI
fmoGAkzTYeXd2San0zgFnAB5m3/QN+rjUpYHALrLdOLIhlcdYu5dI62NnvXCr3rE
My/lDZMw9Vzfh8fK5TkXt3ZLOETiM4n4R6a+ohbLFFCqEd0M0098oitkjdv4ABN2
EtuO4n+VEVtZjVNXv3DJBDurKXGgKOBLAPMvXocqwTBxciU5f5vMUYDdycAfK4v8
hi6i7wj3CoK0dcwbL349wySiW8pbetLfrm5PcQ6+3jDTrqODaBbgF2Dxce+VsMsC
Ljy/yRZXxL5g3SiO2FxC14E93TZZbtLTwp/1mcP3fsJ/y3+kn9qbAxgFP7orlQHn
u/wiFwDXc6NcOsumTTXbZCFikkuDgYpcVa4o9ERsv1gbhTSjvd5GboTvvAFk1ckw
tY3yikx+RvonhNhMbM2OnmbqqIbAVFMH9opZHmgGpcLKFCsckZdpYjrfXhr2Eygd
YlMxWn5CL+yC5QMOndFZPF4Yvh66fPVMo/7jnVu6ylQiT+kMXCck66KeLWcDWc9h
8VnzwECUABUQMLCZqQPU/GwHBM8Z8nqsI9w7gAV5vmHPzV6kEgqQBIT+dVQOwM7a
pRqidjEBnE51YrnQE7ynCZLKwkFLpp/mTVVC+kf8b/tQd4DwqWwWPyRCt9LoPVQE
GDGa0WdptWAJFN9+MbSQP9xMt48SBAQ0glkKItVKZ//hJ0t7GK3GqOCRJ6NNMcGH
guXzDsmFDa21HyrkAR+GcWAaG3LQ9Ah9XqEbiI92aPPaLS2FA7AtQm1IfUGkkf9w
yy72IKzNoLDX5RhCG9Y8WEBVPEnFzjMfDSBtoqEyBmkrqkh1F/wOovD0IiMMqv9m
NbixwpxMvewPedtknQ3Zuz2aspSv8Nv71HOsw+IQt2BiJ3XIBknFyFU+yrG44Hen
FJQ1YQRHvVD/hoPDER9CXv7BpBQYUW1sxqbExaJBszI+hXEoj/GTLFysIwqLLR5D
AjM8jun5NbqdawNKtfRsx983tuGgSYFKi3e6/rr7MF9NBgI1tkIzzN+eSuJmIlaq
NL5Rv0G+Bd3vHthIkrBrYsWvZ0JLZW2ycJkO51FoCvUJPKCZB9CFwdEFSctMOqNI
1zBQeXuUwRt93PmHoknVPeq6nLmPOip5x5gsKUXL7KSQG5IeRl68x7lA4nKEDpsl
+7IFemhMaUDH7/69ZH8klboRYMeoVeOat6djKeJShe8ugFlV2QKr5QXwbKPbb1Kb
5jfjJNDCK47982/QbI+NkYjWnIO9ez1m1ZuWTo4rkkHJs4fOVfLDeUiitQO8imTe
Ho8aGL9aRyF5nR5cd208Ltpkt1QZn83R6h4ROuBlVT4qg7KUa9C1hJ2MilkxRGrK
LxJzAgDMJp8bqPyRWIe4gm7HmL084OPjYatRJ4ILrtDpwXc2Khltca7AN59sNfjz
iX9sD4TZqjy/4Uu/zb6a9NPgbnvp6yYxlxf42/YL3Kjn7ahN7x1h4K1Ea4QJ0Lj7
ABgEtoonwxLA3/3FTY55SVvvFWjoRAM18JEE3KQ20afOymTFmJtLx+ETJZtLJAQK
O8/MmXF81C+Re8kWnZdlISsL7X4TwbqECpNFneeuKkly8y67+Q43tBmw9wSdu+h5
f3ilgrh9D4/ZJAYX7Mi4qIBGk9lKRHg7OVr1C+bWE/UWXiN1X2ALf4wEMnM39vpi
nCe54Uf4zFNWRG1s33wyzLzI25iVs/pducLIq6GCI8ITt1J2//dbwJHhN7VsUz/d
WqKs4D5wpRCBmkMUW0s5R2zv/Y0ZlE9idKe58rglTlq6ZkNq1mMVCY8wZb7k4lnv
fKyyjwnl6QH4PhFNQ1rUJ6cFyNEIy4h4xXA+cJgONfyI3dqda3njzkWVMUmNOVKm
s6pKqgpb2xskGdzCfKeKbg3by2hygWLuAL8UI3v7NaFAKA/Llm7SVw4swA2uDjVc
2SgaJ11hDUeFMbQ7V1/JbIjTedA6AnZ9z07YkwcqsmJ6xHknOcsLzw2xy+xe8+6D
mvpz6IeGutmLIKAjnB/62TpZhmyVIy2X8R4x6v8njD14T1iHHZflK0VPo9NKa1Pj
+tex6QNUgTN0MpaCPPg9d2UkNfuXufwWnNruYTWfIeJSZ+dOpUgV491R5EWm6i9w
GeOdcfw2SqgiWeFmwQbbXUTY0WiquoU0j0zg+0835Qg5IMfd1L9lB8PU/p5XQGdB
vD3VcO81sTLHHGZtM26xbPdADRelcKrCjgy9j4t8xHP3+m6fejfCFZJKuycG2XKi
7KIVmZJ4HhalmeNeKnPpumaF98jUKm/IQb8TgDDHGj+HyL8bsRizOI5iyPLA3QmZ
TXd05gdwltqkhTrw2m8gtaICjk0zWfYs7DH8gpyTSrkK8L5MK19GoDoUPmDLLNCW
iI8InJ7r0yK+pYuVRfrQ7ExioqQSWEdyvZGX+sVOeMZwXrkdIyjA0z2zRcMdOb51
YMCxqa5fLujzq7qNOT9m8cwGTXqGSOEBLROIcJI9sfsoYfjqdQ6IKlXHybeoaINg
GxAfhJ9qdN9QAGI5Pm+xTtq8Jzz68mrFrp82xa+otOApSkjNZ7ykam8WXHZT2dKv
wC8T8Rik/anQ19UbeUo3WdsYFL5KEVhaOO9yU430m/U43ftKrenHrpsZ15giasng
U/eccUrk2V+BysMipkY8Paj7vAF1LbX2hMZfCe/VFKxoXkDUXOgkoDb3SeGIHIYZ
L0JddoWW+NJ4MRzNTVxTy4OJYxkrWd2BAC80Fdf+slixXcR0wQK44ol1MjBLxUH+
HjSHNGsbn/w+gUhKkZsF4ZtiwSXEHYaZF44OUXiXHOYZ1N99djO54xIyhIG4szdJ
EquJnj37pz66VC0P/ZkKKywPPahSUFAkAMPSBQL3Cp8T+6oog3+VBwH2FzVX6lla
9BfunMcupfY9fllMNyMF4ceyw60dc+xGbdGan5rCRkL/41f9mpPiP8bNlQ2iZlWm
Xrby1ew9LjzpcilQXAVArU77f9qgKQZ071uWH6js6D/qH4/CARbq3gRHxsm7GTAE
pKjUwIw0/pGrbsjEh6cBtap3NB9dQqnYm5khNB1YIg/27abT3uFg5pjQ4UGfN8YQ
Kk4SciBE29DDyviLMi8590o0hcuhUkRUdqmmEfO5EesCZdyCgHNmpswbE1Zlx3FD
f62n7+D6tSP39/P8JNdwV8nk449YigNzQe2PDaQJ/rNy5VugPk5oVsvJ4qsAZ9Pc
xRoPeoYZKq/5Sa5xCEUlSVNYlQRGvOQ76qU6JxufcYJjDKYs8SP20cQZXtwukM1x
WaUpNEpZtUvKnbK+fzio1Y9mXKjIWV6Z+Kn+p4CnIyS/RvefO55cOIi2kELmma3O
OwdrL6g562L+hyzZt4Ht/nwptvrPiwOsq3T9Fzx/X7RnuDBzlVYi8WBsYUws548k
lpU61xM943doIeq0G58NAb8JmhNuczfcwLNjjrH6EaYYi0r1NsLmkp0KUc0aJwkN
wHnlzi/3TWQH0c+IuSsDUWcpOQeNwIYcsyjdEJ3cxGcWZSr8BEaAiwy8uorYPBST
59WA2ptaURIt4M3UVmGU/HOaeFAeymVIHmC2O6hmFMqScsnK7UHKtJ7pfTe0s0/Z
BD4OJJ0LjjiKz2pMHl5HNA/sRo8Yszly9zqOxjqO268AQz+MUHnT8UHvw8AyJzln
x8HO/N3h00mqxVQNZ/G05J+W7YD3470GPZ99dq3v+JZcmR7W0jmctTEKgSZKjjWD
kxGmRsJSHq8SZjq85ELyafQ7VLKRoTRMcMklTolpvMuQMDmjTM0ZClVC6UubmOw5
Czwuk4DPvAuOkcwclQUhg67zMYd3gDnJTC81PQJipb+CDqG7GDUQHu5OZXAOS+qb
Sp4ExaxEz0bnZiBy/H9/YM9vmOi0mjPApE+F6y/8KIkqy5yC2FE0UvZc4Ja2FLfF
hD611D8KX4gwGNGfK2L4QZUQNosvBs1r6AFY+MczBuBCP8+5/ZBYHXVxJCT+eADX
2Xchi0cTsyi0q1tk9cYvCaXMrW2aDF43pUEO949b+2Hix78X6N8Hwri8f/O1fI2b
KOpidFSZ6J9jPhQ/RLyQbHDe00T/V+BUTudkFVuXyHe9Kel61le//vhrbdGqJzXk
uXEmMHnbKTqUkH9YvDLvG+6uOWu0hNBwTSuvh1eAci0/yRKX04m2e03sde6HDUxL
Zf164IjlE+Jdk5XtDVPodj1qbGnBp4nuBx6u7oq8kcPvU3iHWDIbMmet34rr9SDI
bByyq6omiel6qzzQb7lyoLsy8JsJNoyY7ZuXat4qpOcrS7WVZLDbr8yDby7goabs
dbgjMVjNEunO+fNc+GuRdQ2MW2BxC9Mv+h38MICuMiZS+O+7vS0xeFQuGRacNlAJ
GhxSIbGm+TdLM81txnSaa5FD8P4ZDHCx/6GWPJmSvvd1WF4O5MMwuowqZA+LxjHP
BYy8MyL6udHDJjS2fJLJo4+blr1Xa5Ik4zUqw9LQJ2T2wCsUtrLymieQ7Rrms3Hq
pVZkN8XryXjW19xy1lOCioCX/Nhzk01QJhKVNzKOQb14S4G6O04Nh6TU0fFAnOJJ
Xz2bLQip2AaaGfh7oH6ZoBZ1mxfQVTWGJp0wa5tISrCvP6HI5KmpQf7Eb9em3bue
Uc1W9cetbTdTrw1xNyLIVxoGS3IGURIGWdr8rWwQgt/QSJedeBEDTZ/B2HpzoZZZ
Gf3y59RhRK+3iGBFFMokiyGLErznQMs48ix9oyxAw7h+/4/4gPLshqeKORq5M65d
rEvd7TqBI5hP0j0sFVFnfXLyZWWyiBDd13o/StWgpi2TIYI2PWdZ7uRyExCnC+Bk
9rPVCGqHuEz/EzFjGmfqVARZZRkYoz3vg5kud139ehlHrDiBT+HSRMw6BMsL44fT
r7DMAZ9+sqmPiD874RGjufXPrIThFivEi+fmmUwv3MRbK6Sfz7Xw6hN77kC9X3dK
u8r+B5KNwRiM7FHDjq2XuY7J0i5p4+HGOsWTuDYkchWokNoVN2caYz/ef2Ux7dCP
2L4iMQNEpX7LjOjYO7CMxJE9oANUafhObZt0dhYkJvbPdmUvszysbQoH0oyIJg/a
kxUEXaVD8L6hxPSzxIFTuUdMQH/XpgUYLLZYjMpwsY43+T6U4SUw8JvRT45WdIM7
xnNFLfqkmOmKVr0Q6bW9no0BCzXdEGzxoChCjY30W32cRJSUEGgAyzpoJ/vXZ+LT
cgOfYIK8cPv6HrJqKHY55nzpuogzQSQ7bjqugSHRQHOObWTmgnzylGav9Er9O/OJ
CEUdW0XXVIyz4zlfYW09O9LaL3eJPfhlJaGDvtff+FcqPYwRdSxDbosnj+pNHKHw
ZlKNtlSu3hwXiIDwZMopRDlSK/fsFAXX3uBZ4xPzEmzMLKQXoivORIcJGwmnAjTJ
64OsgFe7xrxPKFI9Os46GyKX41UUomr6+GgcoWCN0gCuq98fTqOarEsid05P3+PD
H1DYe73KYGz1TMpKU5aCGJ8HKvFNV6DNd2o5+gbR3aohHnIj2uWGCvEFwL8OZXse
wS5kUILK9XHKDYFzRjEfGR0HdILBoljQEigicOvOJkZPUTqdkBSfXGky5bMvpQLS
sFtY2yjazTYThv5hTSyv5rp06hZGURy7ty+6FbIsPf01jtbuTgp8B1R+lpbbZmlx
GlJe6h2wFVA+x6WWIuB9MmR+9GfaVW/agwA6gnI4kChdrFEA73y3DSygqaElfZ3P
R12m2yAFNe/rpnj2rnBEsBeqHgoDp+qJyjlxjcW2mYznDa4IhVaXRr5ycaBXupjv
IB6Hv2YauPTMu2xUWe32Ci5kQk0LmfMIZkrkAkgRYsKS8ZoZ8UlamvYAz31NtGtP
FCsqNk6tMuLT+Q48gPJpIdxA4NbF4TYTqE6SsGJLoq9iAc8I8sqYCVni1nnJEpqg
LKe2LPdhcE/So3KrxP6iME6Y/mm1sb19KRP7yLRyqtXNTMsYUP1QilJYml6M/mBT
6PjVrXjFOtx1Ya67u30BiQ2InPpLvwenc/2vFMFLJpCzMM/XfhukDiQvuepxywHB
oKBIhk4ZaMEYrXpbGyvQXt/dC+2hUQVMfmjzLzelT3y6etX4D7TcB/qwudWSVsf1
SMsfBDkq380UR0OSh3mG7zaXrud8kdnpIIEq/B+FS/2iQV8ZcEXiwCzvxAq/u6Ns
a4U6xmo2irlVER/LF0EXRRyjY+5eYftFtBJLEhVtZV9Af9plkKskO2JTpkGmzAb6
6FEfzTO2L9kqQycdA/jrH/q9hBc/9gQMDqaxef5sLtOSpTun8Z/NEBELp58aFNnV
0UvYA9CojT3QDi1MfsgpGbkE/yYR74jI+rAYpv0GTS3Se6rz1g3LjexcMfmmbRFh
y8RPcDiRiBmmFr4qxXT7YFD+lOYjur/ZDa3W5AWyOFRuCMkjutIk9INYYm/MalOW
erBBzszygIq0uXF+Nqg904QaGFDyzabujOOazF5Luw2/OFmB9YxyOmDs8JEyd0VL
pnMi2QrOh14Nya/QfSl5+jI7CqINi1YyvA3+VkkCKwxe7QZ9xjkkbFITZGgJ/r1U
sJc4BwOED00M2bgjXJTh+WXWAmRyh1tB4xm7ia3Op4Y3Wl6SgAlnkmHL/qvv1Cjt
tXIV2N3Bzmsk/a5n5224dMh9fWGifAXyKo+EtD08RBy1Ok0fyCNqnsbNAp/F6Zyg
+oajQPyu3E7COy7MQ4+aTDn/BJCHlbstoEfIBvaqYEYK++fY0O7XmOmRHabAECfu
INd68wsUUNifAgggqYI7cWbp9E/SMIxiEoBPHqmUJFInWae1UM5Le8ci1aKteSJO
wCLnoSLOdHTyIQIKIZg4QBV4XwFI3TIoKgzui4yF3rFbHfGxrArJSl9NWA5L0Zxp
MY1p/0jCsIfEsm3S2aEnPdapY1aV8irYwj7rCtG+8wNmu7Uw6p2cCv12HnIQuoUG
PFZLKIiBi4QrWLCb/A1K7kij3hUKb6urDBVO78di+ItoQ+D8U6ml7ICzD9QkTK28
E507F6+hpqYczk0yZ6CbsedJTvA1zZ8SW1qnRFdPDGZuZAO3tIzgl64cU63425qx
Kq4/U20DFs2n1pdcdw/JJ195KcFCb2zzu7bl3SvcvUYPUSNIcP5oQGg5ACjJiBZg
9vhXFoVCRoF/7sxb4qCZHW81y/W/0WDC3r2T3sNfvT7jXkjaV/dIYKcMFLXM7sOs
1tobhEPTPTs1dlI676Z5engHTfVj7v1lTd3Ewpoh9rfIAanXgrW+OPpRqWxlo7L8
5mRoWO3YNAcQnCj8yevMmyxaXIcef5FTyEt6G/Kg/4wVOiqfNpY9jqFwDbiGE08y
jm9jXb1Ni6V5oc1V/v9wU0vfVUVn9qmJ9kC37pMMxiomHBXPL6h0rN4fp+mPF8iM
Juz0hny7dZKcADymqPr/WuHG6NJ6bxjglTKbYrn8EaHg5KOLf6+PeZsU+B9k9E+U
Ak0WNHWvvXO0/AJOuzBtpUpVLnsB4CsEENYiv1bDya+9T980tcrORBoRMb5IOv9u
oTWmzB6sLsNBdTotd3T3Yo83CKztfrMoIGvGiNZvUYO3my9bD8s0XoZ4HJUGOika
9A5OSuKNfNzHVyvjxvvBcDv/DOOW2gIS94eu5tf8Bgfxmw/Ayx2QoX/wnPQloBAQ
4crj2Xwevi0xrQA3ZtM+jS5EmmcPTafAkOVhHA7WIwh9GOLhAcjghWjGEd4xKVK8
kINUcAucJ2vbgq9wbWg6ocJlXyG4kjZ0Azg0+nK0o9lM7lzaBwVtacRiQpNp/mXh
KIpQsNXopIU6eugc97jJljwypKOG+Gg2xApk+6C9pdrw+uQVduCuMManTJaZRYTG
KgicI7xLVXsrYxQvrGJ51nS3qCsHTLgszkSOXIXC7lMW9HaO942gLlBSj8BQp0Dp
dptOQINLcqA0H1Cg5uoVPgC9soaOTnRhLGfgUOXAzys3ODUAULCd2EjCd5yVyM5X
qEfHpn9QTQLqzH7UGqcWUpA5BGhJLwfnXFfc9yHWAVg2TfPmbx1xJBLUxDFcn6CX
arN8ro7FBmRMP2yoTmCQhOBVnOHr4MqXMztzo8ZafzH9v623JbIdPAP2aMtG5ZIv
sm61zjk4vD1/vmG8wDaIaWI9W5lFqEIIeuFNzeb9N17oCePrr/tK6jB+lSAMXpKN
uBdX8/d3hWPSeEF1hzFHEaYTLgU4cEjz+jp/oan4cBt1BFrh86ssPX/NMBUT3/Pz
b1Xy6S1jQKu2IS7lOAMvNc9KgfGZM5zjmEb10HO1UwYX6hTPHqTOn3ecb67RQXsF
HYBuvetBH1u+1sth3eOU6YLnShWAHcoxq5LcQRV0CofzD6Gdzmz2axSDgF/wOjDi
4TYEVi3EuDRLg0WwUEQvwoNWA1ot2+oFIOfe0jT6fQJxOYoodhAoTBsUOM4J1rpy
HG03GS7MaDtZtdAu6A8QO933EzA5iEqGB9TisISfKedFI/s9gG1+EEH/Fe5iSqsG
wxkCMMegeMVtbaTgkEApP+wwk/Ra/nnT4obRkfzHZvweeX6crX7sGMSyJOqIGy6u
uDtzLZ+ON4dB8fNsc5hV/pkBRqOvBngN+VWn31h9vr25TS9BhJovF6CzT90gF/ZE
JE9A/Bb+0LkeV12c7omo5F2Soujq0EoWjlfzfuKIwbAB7kc59RPLWHhKPn0iszZZ
zHMQ/O8WWswlI1IVw7TjcgmryXyDgDC1Vf7VdjrpmtjZTkkpBjhsyS+MiQc+waXY
t3Y5JYgzCm4onQiOTKR+QN+4H/hI2a0KjF9vE8kg06Ct32IGe2KWNhq3ndVw5vO2
NpPPIuQrGkWHQ5st7WCd0I6xwk+039PptI4kMUmghi9qHvuyGKWFHSgT7MJYjgZO
Pwn+lO9wxqS2FcTCHPWwqOnA72B6vtq9xuwFI1BUOCgSyWmHv467Vt7mYqqEWvG4
qa+WJUzWOKWxMbQMlDUPKEBj0n2AoMnbymNXqoCJmB6j6K8DXQV5nhDVsZLSPX9k
t7St/p5/cPx6TP+QiI/Y1Lucr/U6eOqNwpFVVeWXlCi9wKP6XbssMIdfVtV03SJV
ZXNvSb5yzT/nEYPNxQUWfsA1TUqEjqTli/sjYhu4WUDW2FoGoTSgJDN//u01RHkG
QUYjpTK/6Tj+tn3Kb8oopzBUb4x9nSnevFCOMGHPLB8ga3Y+7Rse8s8UOlRvc5+r
g210lpvWQC+XOSLPOhM8ohy2tZr1hnGWrmGSpU48yCQfAfP8UCiMkROH2F+xNZU7
QvZnSj8zE/uDOYU9+ig9Nr40tJjt7/lNhUHVRXBbucztX75+eBmDCrRUbVklw/ny
Nn8YcTIUrB5JOmX1jR9rUoMSv71nzZdk0hQNYv8xH809td/libHrc1FDTjt4Mw+S
b4Ij7NfKHjTsPDRr0dacSojpAXYV8EO0/z0Qew5CbS2S0b/fNf/Hg16Nq/PjWBSL
GSTQS/GqoGVxAqPEPCj9UyYGSY8CM0rThIPBc94Xzja1fapuBgGh1+TD/1sWInz2
evVGx1TkT4C58zuO9yqPDR2uH+eZCzGc3OjmBJ01XeHIaIVEVs8P/ZdSz8X9NgUT
h6s/gjOy1sf+Q9jyAAoJyqGBD1hzXSYgGTI1CrU7jxAEEzc2FDMrB41tjZXiaCbp
s/KxF/GAArvuLxsRKQ4ORMquFFvi14BeliU5DToM2m8Es34sH5hEhz56CGumNNQ1
rp1ygAzmk6QvE3qPxTi0qdT1tgNLcRrZ4JPfZQKWl9u+AbC0vrfQYPzji3WO1az6
FVLJc91sL6uLZ0fuL5A1nEOBo+QL62r9JjmZNVBKYuVqe0V1mW4rYIicNIJwAGWB
CvwoBVvxV6MsbxuoiY52LBPbgLZrg86QqdDBzBTG4ln2iO48UA6NmTAqLkI3Zr4d
/HspBU4tMesJtn28fWoUCqHcQlgN7CAdpQFMXR3cxeiDK2ojCOCWGtI8LVUbjzW4
CIpyQ6WDhEU8hhb9jCQjwwEOoafHLbn+SBfrnqHYKzXgEgwps+nNTB+qZa5bmqk4
c9hWoQ8YVP/snJoP95oJfwef+e+rHyaeluGeRZwizMktA1OoeVEHqQ03j0OTb1N4
ZE7s5EtKlhETCtKPS3NkPAJCOiro0m4HoFnpV7wCIevKWEGkHEERFjsLAAtzY0W2
QZz5dLctS6AT+OdrOZ6UfSj6qWnpbGu8F2Gr8vLovyLDCXqQULdmG8q+9XYuL7wp
EiVXl0FbleArRFRSL2/BZ9MDYqQBg3b1RnDufZeoKYXuPIrWMJq3uRkCmNN/btMH
Wgdv+ZZHV9p5EDbkv4FyC9p1UPs+cuPdaXVgAUOYHhM14SJUogHqv33jnvaqJRTp
7y91EkPP0fUYiwHGa4N2GtoJubv2J4RFjx56Nn5PTHYaYEAXCgF9mM4bgaW46ZJd
PhaQnvA/Rvq3a1Q87VnDS+zBQ+lO0QWxD0Ca+myD8ECg5mqw7eXu0F/9Y0C0jRyY
zImx3VUDRrzIX5KCWGL/31ynpp+KaWAgl9AFNHmb9Z47XH7gNLPBqfUkMG4xpuDs
ooFff4y1hlUgBnXhVJNNxd5bTvI3M8iYaejmgoanRazw1knjJlBTzc1FLukXUwC3
VwcHTwnvwfkJE2FMfp8Q/i9FH83Du+EhHFBfBaQbBRT+c1/riQRb+PnsVtsl7kLP
kSYXLbeQNmadF26KOhwll4i0cDID3UNpTppBVOjFqhl3K6BFUkXWcAoqsdD9dd1X
RsAJeM17M3oGAf43ufURLWnV1Loa3D72e+VE6388dPZj8hwkJVRbLmaZoKHrE+aP
4LBIduRki7UCfEfYwlAKiwXl3VgJKr+Ok50btTKdUnM9mDGY7Df9xIO+8BzypRzA
bjG1sM7DDK9fKd541vNBYQrEzsOpitqb9ZEYOPmfdIsPQ6xYSba3Cbu85PJOT45q
C7N9LaorO15fVtkIUsR/8LTpXXioVrEpPtstjvyNaCLXH3Hin2xpQlT8abibocK4
Bds91G3UnoU3+HoafY3K9shBVsq8a1Jcic3IBUPsjPOyzGA+MmimGfQCuGZfgBVV
uyAlDcTQGGVK/XTaQXJw5NJ+/Q2tNuI2Ku12wBmsNP5dgdqgS+1JT96JALCHUuHC
Sai1TRNGHJtQZ3NdW1KYT912n5S/Rae0upNJjHFYGQqGbYrx8fSEDPkXOUF+7R81
+Vqr3OjoKmGwjidlX7X3Stk6WPjz3ynDOlJ+uuzOSUe+jjDUTpW2ID/4uNH3/DH0
nAfIsSDu2xh2+2vz+eH1NTLFJLQgGjDxffihXdj/aw6hVIGXRpoAUxEnIN5LcbOP
MwTnCoFH03xt/2bl2hJC/BGYACtIy+dXfhlIVWK3n9Mz5l1bMbCuGItNIKdDVs/W
0MnkszKDps81R8HX1WZ0JGUPOVHgwCejAQPQd2wbR98bmXYF4SLSf8qQSMrDREfw
YqyVo3VYKVouvfpiARNz2FKqzdqKb/PXnVk7s4ap5/TgLWCuRqbfBHAnPHIY2WBe
+K3CCLdef5HzRQbhjQOHDxd446/uJRfbTe6D6XySebwNbFMgrHe0QpPe6Ai9YEz/
79/H+BJTApdOOKzyoNc5WNs7yQtXK+NiqO50YKuK5LYboMKeAtl+w7dAOb51co5R
2e9nQnz5g9Hxrcx4wmwXOcQe7QrMC73rBHyu1uLgVKaw5yDwKZfdArJ6JzYYQAmr
3jcvB86+TtCXlpqywvkr8ypPgWLzQlZzW5sVAxe1JF2Ak8+NB7JtdfXH2b5URrDB
LuHvR398uPnG80lOW0cqmyJepmfrC/+FBCx8siUgD/J4LDBxUZkuM5t3ZyHoRH9L
WgZO/j1zwMquXwNCJ365Q2tmNa1QvjrwZK6YvVqnmubqqo/ltTLAHY0Kjhvdrop7
4m6NJK7e4/EqtMz73bQdO0gIg4Lp87njzqdqJjFgIL//Ax/1ffTg4MpVRPmuqmN+
reE02BkjbHiTjEfbKUJZB5rL8COLoPdyIVhXRkPNpm+tG4SRGoyEuYGGdV9+k5Sz
HP/EP85+I+a5elKeBxJHFQJvl9BHfczIVkDzK1soyOdokZTle7SK4WaxFw7rb386
3AhuuOq3udhHnMiSaLmJ4FavgLhl8CWNy0rCoYFLC0rRDHRDNIOhpDyIvXWZ1Fde
gb+hQldVFp+VmeHzSb5yH/0fiKWKhhhOS0B2hPfLuDmOvN6MXAgCD3WpqbGKDNjR
TkR0f251iGL9PvNTCRZzMvInrD03sZ8YLn3lK159qF/DyQ6RsPHf+7A2qbcODQKF
scGynR9RQq0SWgI+ZaVE2y2piWVHNxV8udsIx9OVLFkKlas47jbohZ1uvVa4gm3U
MgbTCUW4XCZ7pL+cCvc9kyWSwBVflKisG4DsAtZYe9nf5UddMR0JOR2JU8mrfBPP
0UCA6jqBfjaPThP6SIiOnTxF7C5d3M6xNXR++r1A6lFhRQrxF6JeamMK2fYNK2Jv
QX52gvnRQV8zMX8QtA7KZFZj8bdNAaSLVvynQlOuqlt3HyZn+LgYFa7HpvW9IXK3
T7OcgJGULi9nWHW7CW4VgAdiRBOCtQAbT1RAxr6VvADsPrEhAQHexu3/0eXux1eK
ONBNlkgY0L0Q/d+XHH60/3vclvABcH2eoOfgD5y9/DHbe0sTSh0nIz4RbCG1eAI+
tBmq22r3OQbehSoqdktWoxW1m5/M/Z++wkkhZtsgBko5Yk0CvcwdljjkAr4mQpjA
UlLkoAuLiD/mKbuEeNsKrSq8nq951KV4XeQLm0P5QNb6y3O+TYsgd6skyG82TR9l
rmeFos2uS6e7uhnfG/fIY8V41+Ss9qbQH/njW3OIuV2/qSWRk5t7Z8fqkGT6/tEf
21ty+MtFre5euY52hJst2XWUNefavLpybn4qvX2QwJxh4OVhQsshbBYanfPSrJnT
M+pN2OLxe6ZXGqDeyguMM4YUpQxLgh2llKit34LJ7Bua/iUcQhkXC172FNIwW7xj
XbYnCreMW4lbYHd/mZ4YsuRx1FlvuGANPyQ2A/yBASFLDFFI9vHqsjxDJG59K7Nd
4kt6ounubvWc5p7gbpskTM2F7iOhWeHsgVNvGRphx/VcJGSVJ5b8rycVTLbliRQj
EigVYMo4DlBTcGRNVKnHxLo6uJuhwTxlB9XmSj6pqIEYAgBW/NU1fARa/jIwVT32
T1swoYkLOFX3eMhxpMhNVYYj/PGFEBq2d5SaM0AK/xWubj7ure3qNiHu3S5Pyx39
a3trnY5YsJAqzTd3LpDiioX4OkTOIGDlOwAobuTxB8eLnvSkdNLv7f2WSGtNDmWR
4x+edS5kVgqr4PB8W6SQGGDLtDTg3yOUyHvhqHM9rvFYpBRBPHDESv+NCMLrtxQ4
ZLZ4WlCGUmy3HBfK3cGJ9P3QCFdbBZcyNdAv+1s+u2riA6bb3rECMNmzYdmE3OSF
VzIjWo1PYkfLrlibaLFrbvGGOw4Alp0W0MU2b54hy1NjQCgxaPTgipN31LPeFMZW
yv8OrXoUyHrqQ6DVm+0arPx4+k954tmjp6LEDU8vYbEU8Sd6pPHKpa+bENso78Kq
uOmR6pQGAeqvCLV+gPz3/VfIQY0iiQhAA0Ngg5gkeq2888zMyp1rCUG53whVvdtg
GfTnYcEi3RyiKYLCvUTRl7rEabPMJX5GxeziNd49i8XDTTQDmAvQ0kDqcEi4Bwyv
XFR/9LWpPs3PvpozkifibY66HIO7d2oEicf8Pp3OvXSn+sXWh+/6rb/V/QNKt2gb
9JcsCY0lqUWrsAwbRO4S2iwdq16Qhl4HYOBB+8stvdpgO6Ah0PTVzeYmaDz6Z2oE
h4SBYJ0z/VeCoYY3pSxgMQXoZ8D83ZT4UyH2l3UNO68nhLO55xQTD4dMnG+nhLSi
yzuvqpJIrevLrd4/yv/GrwM5FuNyzW7OzQuSIXuAEYP9aRJ3kk4Ig93MH37eIgOA
D/vPlHyTB6xgRqg+h9ksg0S0D6aok/mO4KhFWsnFq+cw0NoTGU5kYsbzWctxpq7t
2R8+xArs5fK7a8Jo17c8tQDbDzBUAMOW0vtSMPTtrcEnYFXbD9xk74ZeLDguN61J
vcHhOw+BjU+3kf16zVAifo5hyZUw2/zV6GHLf9gaiPMUId1jv2AJVabMbFfRmP3u
N+W9RCZRbqU/H4NVvsH0NiXWgvk3c2llDJcniGBlSKoNlGz1SIduFoZ9Qum9o1Si
NYu37k6BFW2rBM5NbmmpoWsyDjjic+CKLD5uvcFaeURpZpzRa48WJgJWOylUJmSq
/YaefD4COUo+iDQfsOKHqleqsaqN+ZVu53bjhxObUt5d6jW1qCqkPXr8gcMxt1Yu
pxYbe/GGRbWYijpuKMPcQtj6ydeAgxyyjfLuwwJLrhpRJqYPIg92aVgE2EfWuiPt
zqe2DExBLPkEzd3voCz1Q7/epFK0bkpj0O6Auku7LNY0AUj6gyHtE52mtNbRCEo1
8kmvJQwa6VD10ESKk3AIBaxsdo6Bmwkd/Z+cmXDGxgA0q1mydN+aankZIaPXB+0c
fdGx5sehsBGGx7yvJAN/pBr+mnKnqy3wX2F7n4bS1erA9m/xd7qaiEGOU4hoaQXf
ezOJkzuUZgR5X/Mpx9VDt3VZosy32LVoocsEwYVxEGpOegHlhzY+cPiFoz9zCfHp
M5SQMOpauHn6eVoM7PtyxjRVBgv1ubRBpfI9q9JdVg5EVOcFCWupzSH6cGKc1VMT
TqL/G8G0G2XaPds3DE1XLd7/wL51mk5s3KBga1ChymqgIBORqVQ1Zbsk0NTQrAo4
b5DCcCQYW9/CMZwG6NqRvNtacwZqYT2LRFsiCszM5lUnbsxSviTKQJytyMwCFOh/
ysNkgNCkFq6rae0PE1/tSLx1c2lPsrd63ZCZneQMm81HHWpyQ/5et6P++ujmnrq3
RH2/TdErjjOevrHCriOC7dHyxvJ1FiqG0M45SlS3ZVhwRAJ6RMT04lBrXNLRkjKt
LLzbb+moHLFKqiTYmX6m9Wl7HScSK4mXCWO4mY0Q5OpRCD5CpC4qmzNhK5x1aLVb
bAInlxGI25SwOAa9lUXGVlV/BcdWOR9XRl42R8YurjYfVnXMfQlVLS8rvPyOVUGR
06/+bdt0nMYgUd/+l458RDBXnLKTebjM7kaf1qRsY474Gq0IzaVY7eDWxyCPl94N
mNvRkemh7qk4iD0WVRDvzvpHe0VWV9V4yacCxTsMB49k6h+VHPySqjE+6SV8xgrY
4wEYrbpKAXKOiYGjqhfJrk8XCvVKaOnJW8xS5ZtH3vkB/u6gJT4ewmuGX5rYuN7y
Y8G6neeQPam6fPSPOJ0k/QDYMjEvDq/tuXnsuJ66CRYQDqVlW6jW1vHfynfhi0BT
yMesCknhKGZIp6t2jVn3hqmN45enh7OWrA8xMpd6xldQ+Re6QOxDls9vOKLb0SLR
AI5qhIh2o+UNDxYps2w6gKM3xtEEAL8n1ujmQK6F9gonamRum2TDcRW9bh4dEFJJ
LOLqWpJOivUWmuhmoHMBw3dkKpSxTj5QQkK9iNdsb86PVeZInyOvQFUM+GcRV2ut
mfixAQwOwHxmfTEDnzpgEErMYxTfGIPvrFUXnZUStAksXwzKGxbkJzRE49R4FFzK
WrHhI5v+wNucNHp7mlVvVETPRcBBnp7o8cZRsj9uMKN9wGBOscOrYsIn1RwOOVeV
eNv4BughVwlQ9RteFyQTWPXx8vWmotyralzH/N3Oj8IgFRd0D9TvBoioUIvKIT7Q
UNmOY7YJrNUnHzFkq8VWzYJLq8if0b4tDNP5WsHS1yRy6iM3hFG88Na3iUjzh7bq
BhJMloSKiLNelZJLaBpWQO7wOupeR7w0wQ63WiZGZOiymPoGtsJEZ/bOpprQFGub
C+VXZumR1RKocz69pq2ZBnPzvM+bDrT2YNiiWqCedDsRY8ThMNjoL2BSo3N41xeF
yjrMrPSh49VmXQCzTZ11NEtbBaKeiTDwkaJrof33F1Cf5wyqqGk+t5UCjMSBZdAu
eOkGM3SeTJqrgUGvPHZhzJerZ+ix3zuFEDlkbUoL6oJTY9GoJz15zDSLu9INsz0v
aPfbsypFXcXpqwvNfZd8yH2UNE6EMMRt5/yimAhAEXU6SHnE5hko2vO5zW8PL8L1
ZmJrGn2kE3FtgeR9o0B3kRqdtBcVB1nDAVv+0l/wYtqHihZUUOGOfmn+PNCBaCDW
yhRGK5Rh1j+D4JUwNAmlRdLZAup3He18If9O/4IeF+YJUlvoudlFdEBBOBPjYyRx
GuX37I7Zgn0Tv+wdsv8iv3YaOS/XjLIfCk4kUpLZqn2q0TmGWfpTNYzUDrsmZ78w
PVHyoBNckYAsO8ctjj/9uWE/OVYU5JwiG06Br2X2wvgr/1/BiYVRAwp9U34d+W4y
CBqEPKBqrFXh9fGKyYPfi+S4DAnXuhkGVfKoebv6lyZOS58MGegnkqijo8n2Xk5n
3r9RBCigh+wj1A/RTo6m17j64Z7juEF41O+OwyMSJZeTmnTG2kycbnYtON9Lelrx
HpLRssNBUfPvk3AOK70h3pVfwQEP7Rdf07LU8UW3Sey1XFEeB8oSe75D+j4QY86i
9k77sGfpOEur1Ort1CGk8MQ1YgnZhhFIOaZlEWFznUpbCi1lY/dLfI/zdUN5FggP
hM61xt4DnKvAX+ubYWTTrSN8jP5A7foIphQ+Hn6zocerwdTQ5tzHbWcjWYUyakU/
btStx7uFwCAftlJEv+830uOMLJF7SHOCSUNzhpjI0hD5j5QG55IkZ6drnzrZbmLM
gcZbXCIg5CkAg8THsXrOMoy9Zy599Fum3WzAr7M2GxHMV5oOKVviRQhjDYivWOu+
25Cwe0ZEEpXXOk5JnWavauo472nwTja8j3iYt8SVZVz8LkPnrzlW5FbIFda+hOhh
4jUodr+1h01y9ZgexZ6DZL05JdM36y5+XM3xNiy1qWacGGmWUMU/15j+Iv12N+JQ
mQOiNdymRme7ZbZ+0kIkTpuT2kstvf7kVn75Rj8TdL7/BVQnH63x76AfzWRu7WWD
qFw7aOaDNK6MPfojRT5fpQFySclF6+YXvxAbwKo3rRvCcCNDRuDfaoJ86jJpSG+s
QD/9Bmqqc4Sx1cE+y1hZgKKY10qs3TuvavzAYY2FjGj1pHnuwIdPBA2OMxR/p1oF
oHYXSVMG9e9LHfFc8H+ZJM/4JjbagsrID+dW08otJGG+Bo+G6q+0AQxeU8EXf9H9
zNGAQvdbf0ffiLajm9y98S7tNdxPbMDBoKpsviMEy6K0Brzp5TQ5CFZIpF+XvTJh
TDpbqf/XWN5oEKIP2XTk0m4mFKUmAwUM8uxrxOb7XUYM/VNHUOEMseWPDjH0qM4F
GAhvMX9S1zjHwzaPFdi4EOLzJmIu0NQPtw0oOq17Dr09rLyd2war1l06qGsXGgHt
lx+E60Mk2NWuh758EvJR6abUihMCk4OsclLQSez0f641rdgmzrWRjVcg6i1Hf7Qr
gBNhXlX6u1zfVrOQMfYqrU6m4rGWKrAHOnfzEgraq2F/y5pe5HpvhgykzF8eTt/n
JxDSUJ6ALBjJ1j0+EgKlPCt7+S/aM+m/h7HxJdLX0rzbPSC4+4ZhowVQ1LXcM0W+
oKJKVAhNS2Yu6tO9k/8vWD1soBqQuHsJ1JD3T8NLLnxAmUU/vlunobB/4XjyK+Av
TFUQHa21kjKEUtCTOVTWgf4xgct5VaHdure358GQix7QxWcmzX/ukq4nuE+PAwQm
ZrH/aD7LfBBNnMEYwMbLjlsjjNYfW/nLru2X/eqzIf9F1N0XHiN6dEIPf1Y3HdZ6
t3SgICmEbuOob4bScTGau/sw3epbVaQnaIvSHbjrF8K7W2BYGIxs6HWyoYjNQp1Y
8cZg4bbyVJldiv6AHYvX+6NLWhT06IfhhjHtbewqrCuGSefp/gkk4rnKf0V7+8tb
uIlyCzGbqDqTJbqqHQn2nyHa4NQiVOrLPTO8o8mNi9hB/9+1Ac2h9ELIik8ytkCI
loLmG1XrU/aDcc0X5AACe2flxXLDdE0uOaTRCAinYZFzPywQnKf2JtKUaaCBykWv
CdfJsQG5gUODJQl6AqFfsQin+zBMDY1T1xDh2CMgciKbyac5fah3Zv/jm3kZaGl3
rVNm3ccUXCAWWhe4hMDtSxFx29ea0OYXCHu4okdPCiwelkBQTtzeko8+ZyK9kdHH
ypyIg/etV4+gD7d4Gv2vbAhhBYfd21NYvuREXZT8f57PkZuNX1YGvHwgh3SIIwmG
OACpgAbeYD5n171+R+Dh7AuTdiqIOGT06em9qKwgT3a30IpuS02Ib8mUArE35O6v
Ju2kP8A0v8CGXMoC1Fvb6gIhZFWZQBcvWFCz7WwSVnmhcv11nyIWKCFh3R0T/9mt
vBKIHlPY41quZMAiU7PQ2zCzPRyds48J9+y3N3QQt62SgIUwLXmQs24+UkwTQdPm
/p1OLOacu4OJsj21EvYgGeEhG9xjuX15Frbr9d7t/LZa/PWDMr8GUZ6cbbRs3VPv
KFlpevJOJxSOSAbcTm8xF1CwbgIaCJ0/VNqB3FpTqU8+EdPzOsY29w41uRN6kXsr
vjLCqVhw26hqyfg2WE6VOWUVjLHAtPBQ62Pd1mZ2F3pV27svxN1mwnsiU7az9Rqc
hXgtUgaNMhAuwR0DPpg4csSfvoOGeKsyBHlIWXE/VBacPVBsyadqSpO534o4BeV2
Z8AUEl0B8vGwOpqNiRxtx2/dTovwMgUfXnnFl0DelZ0dhzyX9SokDt0jEpyfXq4k
HN6x+7OPPoQDGccAUKpLF2ca/ANxLt5fhKdqYRxj30pdTScaG13ipvYI5nUEs4P0
MB7WYnCSgoQh5h8bUzqcwLM5pHzm12fT8iS0toRn1lCThyTtD109wW7+jaIOQmvF
KU66oHA136wpgX7kQsqFlKH1TIgNRfHXcZkhJhL0B24jZRA99s/oSxXlDeTdAFn1
TF/hHS4EYEwvjFS19etQxVfShgZEf/At40FlDPcFBpQBOzNpT/WcQ915YqB3dn+K
t+0cBh4rIQwanKP08QZIZ82gzqRq7cPCJ636PlJkfaxPyYmlBwKur3Zfu1Plb/7O
tdhy5vMRleDE8w==
=8L4F
-----END PGP MESSAGE-----

-----BEGIN PGP MESSAGE-----
Version: GnuPG v1.4.10 (Darwin)

hQIMA77xv53FwjH7AQ//WoebVxsSONOpMyDcqZIQAJqqt4+YV1lZlfJtmINOaGeO
JjzXM2cOLQYcOwtuEjVkpcxsSsDZ/SeTy4iDC04mMj8ciT2L6OM0qdWCQPPMn7E9
QJhqh8IraD1JsP7hjwjg4fD72itdPTRq5b+rmCMDd9H7vnCVhYlYJ5Zec3CV9N1C
LHktkJPjgFnWk6AqVmwHRPY+HASV2aGm5l8Fg4RdHYxMGSQJ8tma5Lr7OqU5ahkM
joqFFT+bq5/pze9isvfKy17+NFfFLMfK2LoC9g7G2qJDrYpbK9Kpjozfqb0O9Tma
SrzYOsulcpgIbQFYElUT2/P/I8gi06WmWn4FPqPCFRWFvLi1Ge+w/bszlI+q3NLg
0nBI7l2c2jTw3PKld/xtTZ616mLxbhO0x5ApW1DUidVWMSOiaWStcrXkJcrt46gn
uHgm4+vvUaKqk//+C7fTJ4e03K9h9E71C7uT87Sr4BcvB/VST/erO/h8FuIhEcJw
NVf+30MWa1NDANk2BSsPGaq1w1ba2oQOLyMkN9FnFTUOTJ/J50lNih1wjfF8DkEQ
Q5qwKTwyKzSQOTFSW/mJ/iXreBb/xSc+pNEh8im7f+JsMLb+5yi65xjVnGfAlfgM
6NeMI6JyRw24ssIli0Rd+OVJEhEZcTS72pwQl69MClQ4k6nF4URv+WMoU9Y8PB7S
6wGLAAAYGWP0rNw0EgGQ3zaHOkoGgsnvxBjw4gN6QAbnBx1UHG7bQe3DZz/fOf1T
krIHWYhJ/W2iCwSYrW460n33dMq7D+TGBIIobtr3yTU4T5yLHOpD4Es+Jq3Qb5ro
aZWq0md2+I6VNRmczscQvSHQiuaJ4Itj6GVHKAp6TfM4YICNxRBni4WlaUCN/Brb
S1BdmrZrrUwMxW6xBUBNNdhfYDGKJM/CcU8DLQQfwpwjkyNcCFPUYeFV2M5tz38n
HPj/6kg39VS7qWQE41XozYS1LDm0K4lqbSmEcEP2GrGsZnyWvO0Vuz+pl5jO/Sse
wu2CEBdI0AESbXKJi77ziH9gOyyXiSl59kjt9QLkFJMEmUqzw5NkPt0sQOBipGPZ
07kk5FnOKAOYPlJIMC0mL2EPJE8XzuGPrg2lvThgbWgvFUkcLPOl6cj50A6kmmTR
CClGB8jOuncMxKQmBvdwQc3rOvtg3mUZaY7wwAHy3Cw8hFGTxojLqvkdJvoEBiIy
Rn2f6hZoQPYKETKiQoFsOxrFWYoLRQWyL9K00rmGxcCh18ZY6ne+f2tgEu4qAjgC
z9ODmZt3vonMqQxTHagJddDA8zpELzFH6EEa4vijQkMPTRBEkmNCcFrSsryVkv9S
R+BxcHePhHaq2CSlwHnM9IIBr9gB0evYGMdARO8WgNKscneXelMYUL3sOmmFJi6k
ZCUpI/ExLnljoJFN8XEtyVINg1LuP2Yuv7VgLQ2eZYjJJK6OHa9wZo8zzUwxMXT8
/uC+MIGzrjNUbUR+TUBeU5rJXeRcyPEXrlIqAgtIFJMILV3W18BLpE+GD99s6AVC
AEebcyjE4f8tKhqGthK+hIaGrgGKsaAGkp59c2eNuyIRpb2ut4ek1OvZBd0bmQuP
PcSBZYePYjxwbgGtctOx704qXQokSqPQkK9vjdF+a2oEW7I0zHEMrWd2+5xq92KR
crU8Eo/PdF87reOIf2amQv2pwkwFO4W3iqX9bFYq6MLiJS+Ze/D5sp5wyxkRp/da
c6WyGvTSOhp3UT9BagJiESeRfGlXpz2OA4nmtacimU5s4KcLdObCzYGvuLZ8QkuN
6JvnM1RiI/zeGtv+HbFnMpvxgWNPN4UPbDyZZHLN3M2xNtNzdct/37L1OKJPjy5N
0Btp1CQe/dyaqbv9M20Y7KvcEUROdmU34Nu+z+I2UO3Xn60NqQMme/cM5UK0G2Mv
fIPo7K7khrR6SkW5Zqpl0vJaQ2NKTAI/W58SjiXNDRc2/RY2ZIGD4+czF115mNqm
zq9D7MNNbtDArO5ATl+nJbjh67oN/KStx+7WyKlv+usn3dRvxriA/1yQGHX5MP4J
aP/UiRNKZbrD1f3Vjg0tcHh5WftVW1ETumsgWsm2/fzqRv8cWwaQK2aiGtPvmZVI
EIN+u9MzbtqSM4wNtBcIitJzmJnSt8eNmp20PR6Qy7IeC0wfzVtd9y+HynRMFaEa
gRUlDRkWL+5pUbW6K14jDou/uOenQKGWAAiHUiBhr3yogu7GpWNIO4Uzwow8qAIB
3uXePbx77Qet/RK0rdGs21mbwaBT0r7GQEgapZ8anPaZe3rTlyNaRoBCMxtnTr22
3I5dC8MXLfstpiiT6eI+kxy7jndjDlFMSiV3eqm9wGrKGKa7dlFYq1TIzod4QUn6
Iu1eBWZlOop8UtU5tlNIKRuGEq+Zgb4WOgDD37QwjcxIYFwY3ulRTYVOp09x7Bf8
QCXfwRCBfkjEreGvHvUPL7EEPdbxaMpuVmStuW29ARvo7TAN3BNksWNE/6Ew+FnG
OMJT1FJwQLJpCFBGDeFbaKFCZ82epxQB1Z957/c2M+LqDIgm2a8pL3MSh5l4DlGp
XJZJr5K74nKdP9weML+W9Bq9HCeQVl1dm/9G7ilBvLM8JZ6TDEDdWH5tkjQOzJ63
CHIxaVSysLdaNZJLktmbu63iRj5Ag7EE4QDwHFsPmI2Z75xbbDvOmy/nTZpGKWc/
RBexycO/yPkeA0+yLbw0/O5leZZlDCUCPEzrL3Y8CMWIHZb9Z+OiG2riH721BAjW
HpRHdIWZfaFYL2Y/OLlPqxtpdpc1R3/9N/68Y2fEWFbJXny9CzRCndDQ5GJt4LJT
PtUk3iGCYFXBLnlEvXxsYDppPa7ygvzBNLqL14dtC7SnNsDoD6JTpwv0VIFda0pM
Jga9h2g+NNG+CGpJH/oWPExkcDumYmdSGBNobVWmVZL6cqf/qhgP9aICcDXC1zJ6
hoNNt/nSbg/m//bciIwAWCDmtIodiRtpOvCpgHA4cn57Tun1GUfI+buz/siVxVDO
Odw64090LPf/YsxV5s7UFIHXI9P/hl/3kluVnicOl1Sixwy5NxbtlwN+JvwstnIi
IMAGhV7i0JL8HDNZ90AQJvt9PLqLjcGE5zyqc8BXNYz3bzOHhT6Gy1nP1HXcpK08
qlnFmdsRImxcwzUfBiXDh+bxZJVy+N6l9PMd7EuNf+04V6zCgNce036asYl0mDbr
U0w0ZwkUpNT2s8NJPqWTAlxE3Kq0HRxpeglSNrBJlmjZPyyT5cl4IpXRTrOgCaQe
8UHsNW0sRHXm97hj/SzMv+SjcCPFHBD5e6CwOFu59DEv/C1+lqss7dd1t2z4kkeP
2p/BJ9c2axw9PZTxfI3DWlwe87SfTqLScWYZPclvDt9C321XhXy8Zgl+GFCm65Xb
8FBQPw516IYW03P6Waty7E4a/2pWjk4WdEHU14d3OprJ6ZvUK4orPabBvDN+V5q0
GB0uLiBQOmd4EssB5mZuM6OGhx8z1lQ4FdrSU8fgpnhP02CjGxgQ3ONKAdhLCNpH
VF1MgxLOzNzk8S3TkpGhHBWxB0xVKZBgM4z0h/Gh8jxaxiaXqPNOBVagfLZfXsnr
6AwqfngP2a1/B1mC558Ij7otExvuxppcr0y/kdcR2Zik3VZ6Iy83UPdkmgfHY6T7
wcr9czPgLQUo6p4GFskTDc1jUIbbWB9mmjXnoqz+PYh58f4uzbgtKb6Q2tVd+kXY
5ELSncI8maEgBg+VeckOzd07xQ4r2wpbRSZHnfoimMXde13h00XJ1xLuJBAx6D1+
AjQGurUvj5/v3iqNBd/oTDK+fRvibz+v38r5qCkQm3ejZeF2rMHj4i6vovFaZMbc
zB/qrXg/AlNSiKvNxeMCc4P8c8oWiY5lk6t6wltZh2dAqyfbRvzW7rTUnal8V1VB
vCV0cFkeMC4pSpO8H+QFRWHGe6cqO9D5GaxoYG6rEDQEyN5emyqJf2Owzmx2HSwj
y8W3JfLAIshE0S5ucOWgvPnSaxNd7hFq0NbpTSMyIDYLltGRoA2/jDxjozBRKgWp
deGB0sEiTlxTtBoUSN64fQIjonNJU3i6VcO9swAjNOf4nHyb57DPLMmVcEBHYHO/
yQymEugCrrAgRWdS6QdWv8jOwAWUGkh+HNdfgxbES3iOrac3fBxcQLMDaSEQbBNG
Cch9PT0F+lypR9q+YakEMf1VtV5GnrL5lWO8Or2e7H1eIFJRU4Y6U+eshetnMfDp
LioO6U7Qi9NQgJYwcKPDs8iZTFX+zqmeR/mGXr6T2J5YOe8Qq7gRj5ixOrrfiXvO
B0PTInbgb/SXvMtwh8+TgFjz5xaUvBxCuUlUUfNcUDgbEy6uGXq8apgbE0x58k3B
67Cr0YkNUIdmjcO7kR4OXl2nH+H5fcPnow==
=QdqR
-----END PGP MESSAGE-----

-----BEGIN PGP MESSAGE-----
Version: GnuPG v1.4.10 (Darwin)

hQIMA77xv53FwjH7AQ//aVJPPPug6BgTtH6csFzGfdLO+ijgqXB8cqVcxuY8fomv
hHVrmhpVuU5DXT4lRnCVVk0aBcdtMIj8Lszp9MWplAoFMc5Ydso2kO4aGTKkECNo
JTW9p0Z1dUnJLmtY1pG24dNAHK2zNrPX8V2esAeJW/jeUTu/nI3Fd9A3s0w83cKl
U27Zaj/UF4H2Ga8+EFjy003dZNSmjdJmvq6UDCrb0WW02aS28KZ+gDccvpJZMnjt
Liyb+RYT88MePdX/udFXp6N7gMOC7vpoVrKXKjdcnCx2DXS2JBGH9kXJKjcb5d6B
U4kKvMllmEhLjOXXEjQCpD7dj9jbBz0h7E/7YinhbzUZkeUdS33aoJnqa4wN88/i
Bny92zfdLzR8wgoIaLDs3z0oSjE9fmulq6v7/ITUomjTvHoGztWLBmx9hLIp3J0V
Cj3nmC4drZLtMdM7I3HvBmW5mZ6wdw1ihD9Xz0Is4JZQYDq0bHB3t3/s/AquAmNN
eLGV4lZVYDjdyxZWSjSqsd9ZjonDPv3+iyCyEtbTNNrIosYR8yJxX6Gn9FcBllAb
GL+2fiMFIUp4oA1Jj3q3PDa1k+WzZ/GMj4R1sySQxBiHbrYPzxtfPHe9md9Zn+iw
nLqVQb1Gn3GYvjUPgZKICOz4pIXJwBnAo4au005UjA8cneqnqqsoQ6msgmWaxY7S
6wFVZZTk9wocNrqKK1PuLyVRQEkozG/JC2pPfbRx6Jr6pFKZJTPKWO5MmnOPXSVb
hE99uf/YpC+j3mb9tIG14P1EeRw5n86vuXdQBYKo+mm3t57txANVM9ycirFwQN9N
adGPmMgVB2Jl1pkcYSDZ5QYapeuh0bm1rkhZu+2ZCfc10oajQzXAbjythef4oVqj
6tkUX6SCRnqQpkDciRHhvxg+JlDvaNCi17ztkA2sROxSmkxVQDkDj9k68Ct+kNux
Q81CGGWA888z2j4AyS6/BCD8IbT73otcyswh4XDdZQxnbRbmHIQnyFFTsBvUA49b
GahVMgcxwYZTkMWGYwInGS8NVv0oJPfA/RfrH+9FveV0eUy8f/GRO0eMLAAAz3bt
f9u0EkGEWrnEePvveRYK+v8zo2F6it8jYvDEmgUDdit1/TExPIAWjsty2vV+kDnf
P9lANLk1Tq7S/De4SLh90CFgbIIktahbisG3YcmcGKssizWJIXJxVbqMA53ffpbk
fX50gFJ2SD6Y1GpzGL9qTb5R+LyIzAbFafVnx73VQnquu3Xy204BFdfk25L9xcHH
o7Yo4DDQev8ZKmoutTilj8zql1PnDSJ7vb2nNTX8bX+nEh+v8pJyaRnRCGX7W1VA
HSNbNcpusO2SbjCmOCV4l97E2Q/IMFU9XIzSzAic2gbXYVvIZgz9lZTtgEdR2Nlu
2DG6oKSOyWi6g1Pd/0VgKPmTyNZwzrchSZG91hrU/THvV+td5PsCr2cHd+9lzI0S
uHm/6CuD70jDnv7/Tpe9ZNCtwVyrj/rd5QPcAE30IWTzmR2tOAT9XPeUilxuZ5Qr
yW5RCv7yhRAXgYRJB6Sb6gXsFHk9TVa7DCRe+tZhaWLk3cvZvuzcuXEDHp2fbiOz
jQfZfEHwE9aZwkVCeT8m39Dn94J8fk/8iZ0PlwYlBab/3jGMorlI9uAU34R0PGaU
nCPzixi28zrEqa/d9LMjj52OQ8sth/4cJMMe7lrtFjIZ28KO4NrGGJjSKd7BF0QP
5hlh3GYBi+x9yv3pj1WdfzFRTlEi9f+P2YsR4VFTzpETuF2phJNdOJLvUOrMN4Vs
jMLw2aVU/ZvsPV0qqvB3HdL0yLQWOsBMPkkEcSUKdSFzYAlt9VKQo+z7QqkQJaVh
ZqiKAyv2NMc+px1rve9nYaBJNSYOgNz/STctOiNtaoOWt6w3pv+0WnvjNE6Ss5C1
irlyn43Bgc8FS4yK8/rrSmqbuv/WtdXT+HDv5NGHnVmSK4gcYBkFD+wTbHyQnsoo
jaX6Lc+OfbcQSwc6QRlqX9dSkK5TsBGc++8YaGX/q/RHwk9QZYeFlzwjRoaBs5kg
Onw5QpPweiqJpW50erjl+VA802AdrHWY06z5MSLOKMQYoLfiCeQtn0hLlp/LXfAJ
fjtn8lqRM1QF9UwPbgHSvznOjrx2Oy1p2V9vRpO7p98zVBt75Gs7PoBCdG2GSlnA
OmFz7nAQFvIRURFSnwFlrZAJZ4qmwazVvhUFnn8KM1nWzthMrZJImYmn5nA7Jls0
8FYd24DGSeZQLuvq9WHsAOPrrpY7aEf+Z9YyUzQSkP0gEhlSA/WivcfBOkZ8Zo0g
fDWGQOKprsZ6afcKmIsW19Sg/pAWR58qUgAIa6mA5YzNMdJ6jaSR2LklBn1R1vz2
t8ptK5uVQbLkcDI0X8qpLanC/eceX39WWgiAgxyOMMzW8NCOMxa860a3tciV6AYr
1l7BzeC1qBswFykLp7NkDrWY4J6nUqB6SV7qPk8eyyivmS5HruHrWIca4P4GiqeJ
KQ9M14XYNB67UnRnXEdH1jiZLfkai2pLSq91ZfzuG3skgrwK+zjMvDife+IhKFuw
jerJVM/OhbSAMmD23DAETTU+0zqFqEQWCbqjOeMCtLZWDrRySAX3yrN+kp0Io7p2
K4qeieNdvfF1GPp+tVrHocQwKahMcKkEoqKQAnNnMymLV1oyyQyaOYWu88kTHAwj
CR4t3klGIRfsSiwpJwTF+tetnuJzhMppONFX/rZ6ASBmYuKd6rXCETySVU5m/3jJ
3EO4QEoQNOHB45aRQ+iPPeWXiyp2NfVlFbEUaNC4RHYiAEc3CNrjLrMDwdZvC/wm
bHeXoliajaMGSBTYVnmXPaRXFI2RfFHNMhntLKBwcC4F3HMw8YdCe5iQyavMaZ9w
RBF7kwGFyLc9zknj95aF0qnJWx1WCNSr77hybYDGeypY5z7BHEgmayKpssNp0yrs
dIcoJChueM6dm2t+Gk7fb2BJcLLOVJ2vfdWJVUfe9MoEmXE2/sI/bvjYvhMWWHYI
Oy0juCD/j6+TdYdu1ocYzQ4rSS9E2BNwmoCGjGreiUOrOpGpZDg64nA8BYzUN5zA
wZzo5UozUlYvLjDtEwQ15W09Mvc2gH4d4kSGZW4X3OqgF2jLC9GXzrcHU5WpFws+
mC7D3k6DYI+dWkhK1B5wGYdqrzkYhYqJVM1aybfjsf1t2R1tlqz/zXPO0oG+W7AA
dvEha5xx7fEj3xm+8UYfIdeZWbv0hdHNehwcT+SkJH2EKnvWjJS0+BM/F//XBlb3
zWI0S+NRSBQbzsxK+Rqzby23c+sM99tMzaZeVoVhprVh3UsHlPSppDoGb5isGeey
+McA4X4MUm5h2l6+Ds/XR2KL7OnTUMDkBG91O+6sHIxcrOvFpBGCzfMcr/zeRNeT
9Idr59X4rWmH9HeL/zBAgTXWzuAQzhKrurav49d70Vm86euxTYZuWp6O92NUFIwA
TVnlvwhOuYWz8GqycqY+/H6PyxxzTFRsosUQxj22KgKUWqSA8DdWAf5jLgFyst9f
Auk63JNUEISFXeKjPO1SPL0VIkTS7XBihUuUvJetLd+ASn9j6Vq+KCBFQ1fJoluP
kO2uOb8eEanG8n0Fw4tC4NsrQqSDtZjhvR5HwOhqbpjk1OtlDcaYeMjohM0VNPES
dBSinZkYuuwOGrubX9kPpO8AecfevfztVOGIkkEUpmcglJfAZd7DKGuxQ73mFdCy
ZZTZPmyi9D7EyCl2Podorm8FIKhJv7DPNbh3X+Kk57kHbK089CRGNguA+XymW5da
VygaX67JkRYpftva7Xwr+XF8dw9GltRucgIITU8Gv2lHQS3zClhIvTQjzMoulJj2
6GNky8bRlusQcBu+CGHc5cAZirrohwtq3fHqJJjZFgqBj9M7Wote7JjJuMhVFHkN
cg6EbkO8mpVUx4Tayj4C4f8pLYcKyJiIrkCRdX+Jt28c8AVIp4cePAUqx0KrMAAr
SoE/KW8AQJngdz0CfN3przooGys4sB+1BeKU8ih4eWkg8N9rwIRTtWU00j5n1E5O
JI10fONiEEiVt0nlW2zI902U2O03pV8KZT3KF/boKP9APrd4q5NePt8lXrWZUQJy
5ZqIr6mkflkWm3MI8RcPw4T8u3GEinhwdqb3NrrjbW6wsxZCFEHp/0lqW9VrweZm
3nm6W2PADl43jJmfdqlTcFwhcImsgiBlLQT2O6bgPjnAMLtTdqeCyLxRfoGLHHOA
cCJuvMEUh+AhPfRe0523Jh4/neYi7FGBpTSFqC7P4yqyBqLIiZ1hF82tJsXEygTw
xpvwaMWc+C6OLVWb9dq2ScARN/PqCGY8MFt24utPp2xbHnhlT7pB9KUVWneyliK0
9ShpPGC1UlAY+SuYfvk=
=BqqJ
-----END PGP MESSAGE-----

-----BEGIN PGP MESSAGE-----
Version: GnuPG v1.4.10 (Darwin)

hQIMA77xv53FwjH7ARAAgDKy0HppL/s0Ri7vYiWXNTaAChZp/+XTL/52NIRL+ymS
ztnx2+3dAE8tthju4BV60XTl4/PxwL8JCjoXwKQ2oB60MhFqLXKKelAPFEhKTnXA
cZciWwWcpi1p4/mQUvMjDHKeT7KnoEqE9R2sRdXlYxqZoDlj46hOxZvS/6J0bEth
RrFiGeGCBWCkw9rJulW3ref5Mdt6xO+934KGomvb7Hwx2eAjf1U+PDW7SJOzz2kn
Do64oRQo96spekxw3NQBOE4/ab2uiPu8Q86O4UdbUxgm15L/yN0PWgsEmQ6Sud7Z
J9EfNe3WtMufEHgHoztpcWMbeFDtrZXxlMrL47hU6Jbgme/xMcVwFd7w+PLT9N7B
qUK+E9ALb2O0ME9dM0mCIgwt6oVS0nDSo40JXY5l0oruwpwIp6Wt2u4wmbRLoSof
JJf172HYVlhaxZFyNaAiFe6OOAjI3Wo6sttdCPvNGt1CRRVXg0dc1zU+EMo9fkkt
mQ7+6KSH/9OeExYx/rO0bzTEykiZfnIMNg5nRfrAv4Ex1Irsso1oN+ucJhDR7Abn
N9o0fFoJMg5DyszC0wtBIfWJwxbsbkrmTFGNclwiSKXvFt1XZqgPJL2pJ8L1X9gQ
CmiYX0ifILsUYrc6S8i+9G8R9p6c9l85Gfow9oly1N/8yENd2bxoX9SvHaZhFM/S
7QH0il391dGxZaN+sCzssupD2ZZVSU1+ZbeE3Yo4T71bgjnfEiNAU+jc013pIM7Y
bWraVCpy4i8taVcN9ZelSXW55GoqRFPFNnWpVBUuBM55p+5QWOU1DDFJC5I98y67
yl6wDV0SQwcT2hzHnKRutvXq7Q/PjsnAiS19prnUZSoiz5Ey6d6tAWbCRgbzVJDo
6G4xaIZhuvtry6nrMrcupfTO/iIMAoStoFY4LaAT8sdjbtkVkENSkVmRDiIUfVh0
vSdQpTgz+tQnaKs/3SuFJPGPazWkfVdLbAkYSc9xhZvXu5HW4s2pULdJeEeqomDE
4aYbuLeFNuvcuIUOK1ZHQ9viJOGHpBq9a5QN4+qOAum+Egqgqka4oHVnHaV5siXB
3cP+7n+5cyb1KWLvoHxmBdPTO6tuS9hwy4f+u+/He+1hEzxyImPltIEB8FBvrlnq
h1s1qyhsfoySDQPz1VyyVxPSmTLfV7def2X7sxiHWGHeoaFxzZiIJ9uyemGMVRyE
boOEOJ5w2XtY2wtTqKE2hxYkSXdBL7GqKEmdFFEjyQfFtzburyOrxH9ThcFrcHSa
31+Q4CNuVYo1dqlM4kuQkmK/vA+YqLRnwT5eE3/B1uct4ExEizmfOJ7venqawyP9
ep+yHOMQl5itbY2g88a1rz10t9HpLmizfSkTx/0j0QJs11N1ZTZTZvRjRYJ54JGG
NfnooG2ZYo9nJH/cUw+iSbrtOqGcbY9W3oZFddCQ2BVybEOfmJbJj6ftEBg1R4HK
3eHk+JcoUTxrHXo2j/1uYmHy0l5nJ+OYAm6YSDsffSgYnuUTWgDxwOyvPMSGDH1i
H/CTFFWxKQ9lHVgb0EKy2Jp4mm6JaHEPRVp2r3lI/aIjK6aaOJA4kfDfJ6NjAJ2i
S0aDNHYgY5zP7OL0YNAQqWXB//jYv5bNpctj/tdoBIuedgdGPtHhFENmSdbV0ION
YYOahQmkkNzEaodBGTnvwZOTHRqOS2acvbpFLA3bGIAjRw+Ia5vlNy8AMYTkDTAQ
rLfo98eGzYStJqhTAPfzIT98YJNVVQtfNHd7nA7U+NfB8/KZ0PSAlr2eDiZT5dk/
e8VfF8qg2Iemb27efiTBTOTdzB8MhgvirLgaW1xqXsV6IiE2IftlXa4OcobJ44ZX
EjK3JxiQN2EleyAI8uqzjDESvMF0gviN/yHe6PnN35OW9s7d7sIFYI4Urn6ppYrV
H4YYF6Pcl5Y3dfoquS1kXdlCdMGTGfTOafp9uuTcpOqhm5AaDcuje04hD4WVKxoh
VvAYZdx698cxez2+ND/lSCqTaP5HA6P7sbYrBQPY/xPV/RvVM0DUHIZrFWyOANe8
vgLVevdSQEhja8tBo3SZOePti7YhKdoNCZfcU7XtgxYusJgIWE5OvSAhmO46QZW8
0+mdpyrKWu43GxNQkILytUFsnWCd/SpCepLLYfbNoRAgfWZX0VgLKoab1MW6yxxv
PCKwnRXXO8TBwMhHI7lTDk0b62tzN03zWkLJnKx/vK86AYekLelYTg1+t9C35IoN
xRtsoDffOvQcmWHT7LCy45EyaJPLnaqZhAq0knC/IwOLL3QD6CcPrgLnXKo6LmX2
L8i+ifzj73qurwdNnjUb2CVZzSPayNUmCegs2dsTe4BJlznA5dl/XPPAn3LSs9lx
Qnj6U4ofNgwF1mgjeURh9G7KSd5hl6sKtunvjPtslr85FTSuLabo+sh/mlcXhjQj
6pCk5XzwiX4jVvIp5mFHPR9KFn1r1KMShGIxKjgD4iNRW3CNziBBGnlu8PUyiedo
4dJ8gRNTEDWQJHcGBsjxGzTMODn9K2CHTULamngLYob5aW2w1fS0v/C9+3tmCiOU
0MN7kzIiX/1XgohEo1p7macDZq5Mx/2LsuYKXNK5kmU2A5zcPU4Yf2gQ57Pfp35+
vkZiHOcyzDQ1jTqf3ZCSz/ueVj/Zc7wmr6DB6V7o11SQZVIxClRsqreoYTfH6AA+
bzU/zMf/GAEG821KCZYooEdryTqYk6R0rqU6vXKcXzqqYFysH4YzUuVGsOJ3QT0C
KW8jBkeQ+59gAAGBolmYUQkaAbsukUK0cv7F8teTO7LNo5D7OAS+XmyfmKU9HwDR
GH8GMaOzqDe0BrcM6X6uMBmgYP9KP7f5YlqnkIYlR9OxtPQx7avC3qFGzCmoExOA
1V6feOIOQt0dQ0oTsXa4mVe1WF8uXKraydqkfbefmAWV5Qgtt4jCJXsTm4IXp4/w
K1dmsaKFm0e9h2/lyd3/qFAZk56ttyPTWK3z9CR4JgiEm5zPhbbJEI8waH+XMTDe
Hw/clhhuQ8hew6JQsqf+3RneZM1qdXY97bXG35Z1FU8OviCOQgiFrF5zQNmWYgAN
OCGqHHi6/K8NR2tgmHW7aEwkGo86f+bMo6qW14Pw0rtX0PBldGdTObIM7P1pO+KG
1BuHBXvfQZ3tYPfGxfNvoWQ2jix4b8KhD7WgG3e/9pOCaODmLF0dS1SHzCYAXOZq
bPIgZK3osCAbgADTnTviB0eRt7swphDzhCHo3dvr/pK+pNPdP/wLHh9DGshAnTrN
Lgo0Ny+ihBPGjczutiJhtj43U1ZdSzB74AlQpHpmJ34fglwEgQm5KtjCf4RQvi1/
GrVzKwZFSIrZtdWZG5rudOBhjGlseMg9tKk6AIJUd/T5Cw5deptaMSyQAFFLKw8D
hDQdK5KK+0CSZSWub2JCbGGLhvfSz9XHXn/cvpRE9HqDk5iozbSl+Y4NTbuUIruJ
5aCS1yYA/rr6W0Ve15Kdi5X+D1VbD8dRvWx07s8jpepwLGcPzSYvrFbdvRYUw13D
J/DG+rMfp+w43vGXV1TyGtj8wlWefp3bpNhZb1W4JnQlO5tvE8nnv5Hs4fO94/cm
G+OMuhbE8bS52nx9vBEYcHMdyQSkZLXP3C6J5LeYrqQOio7CDLsBZwZ8YrZJZDs0
/SQKRlIMC8enZVlJND7rNqncHfi/uVb8dzMORJD04BlW+OEfaVChSw3A+bokkoBV
3p8XC6ok1kdIY9FZGvmYM/d/BP13g2FMeJipYnnGe0aYD8d///Ct6YtL7xPxGGx+
q8S7vFiuGuQQAPTsLK3mWpw/b7FrBkuUpMR9i+5PTRjsNqfIporTqBJhqWoM2PGk
hBOn13O5s0GeftNTDIQqPBQsTa2FxOpc+pR2IXo4krvRJdId8Izf3R0Hf6szt3Vh
1OuURXh/yE3qy1vwYaXgndGNrp/cRQBq5K/Sl/1W3mBxTBsXBHT2Jke9GH+/wZE8
G8PcwG67z2h6jTgYVQICduKnqReUoQUvLCZtp3GqR1wPKf0c00zSvA4kYCydVTy7
xEaRPQoipnLViYtUGfX/WVtzPRtKcjxRgbT76oiZIohf8vZtRdy6vVHXi8mIvZY6
xA7WeT6g5YQwFqaO/z8yxHK+g/Ljf9SW7WMbh1KiJMMrSAnhI+rfcy1Md4LXDDvD
8NRTS6LtJO6/ilw1T4jlYY/gbnT+WqAGkq/+ews0xISrwx33+SgDVflm9ONi39KW
dYBFNcQWG2ysgv/zqmL8A/diwUWlKlCq6FMRSFvKIIXye4or478dHhy58vM9oOAR
WPJybOP5eJuyTP02DbUw8a5s99LJpHZJkSzsmaAX7pkw8cKhsBUSyQFcf9kya+Vp
gb4kBpeSkl7l4DAjp89D3Dh0Syr94V0GJvJL/oMLWn3p6bvcjBX1EDLFXG8hhYkY
6dNgEjtknswftd7UiElRe0QrcenhtVu2vZoIfGuQd1rvDDRFRZX52z/06kreu8QD
cAH/5oHkSbeqdLiX1zudxOwcQbTAuMp5AyhPnTM38tUZCwPHsHjFCg+5YxTxrCvI
yjqYzeGfs6/tAcgAtOBeZAO3BngMGj+B/X80ClWeC/qmtHW3weooHIGYZEWSUIoo
WNJgwCqS6oTjE9GDCHqPsPVgo+OUwMXfhmZU+xbSz8GsmUUUbK0pP7mIG0bBWyWA
YiGgGW6RzCbHCuWx2junCKAS3vp5Fs3UG1lh3hc8HwkGBLdmZLbM09YZdb1f9gIR
2GsXj/wJh7RTdUScziL3/LU21FMgT1bqABKbiYpDYSbdD+MhPyOQyzVcMmfnKO7i
lbeWcmrJzsubjouhO0HsFx6/zv5ZHsDDru2vhPxHKZoS1xbmWQeuFah2Vtg7RYaF
rrrw1+mSWBw6Pp2glTN0XpQ7WJFCRSDFL+D+nxUFFsyd8FnKhSI8+zP/0+i7U4mq
En3r8QWIU9IOc9blesvcv16Ny8XiGPGkzj8UG01b08X+lyLQ5OIhMAtnnLefMBxy
a5j9T4w8WKA7Dzd8PjInuVruQJwRB+0a5iKrWccYctpo0ve48DYj/rA+IM5Pltol
5F0gTcfYPFpcPg+iIYXj6SVKEkQXLRSYHzdbSYh7ALTvKKe/zN9fowZupJn70kfn
/nMe6jLdSopjSnqehm7/n5ybgrG5jB1MWzHmFS5Abx1cNukqWbxZIC9IyXZxkZIJ
0ODXuz8dSh8cMb4DTwBO+OFIT5EEDKCA7mYclFMynW15sPWdXim+g1+JCDSXO1Jf
ffxczCAR2e2WtCrVp8BMTTRix9qOmzpqKVfks1RacqaD23VYbPXcjfzm5YeW+oTq
9JOGY031a2fBOZXltK5f486zW1+PrzbZT9NH4KgT5GaxjmQMIt+V72USD8DfLZMv
t27g5tudFh2ed7Ep/jYM+kAjzntcxdXr30PWpBKcDqyaUsPB6YBejJr5Rp9SVAfn
xGFvQZzD8kc7ecsbU8kLnlZMfMmT46rIEGtAv4KLle6uD4lsCQIwWsIOgYVsyj/Z
8ibV4jv52r+AR/JHo/qf2BOjuSPIJS9lKQpFbx5tPS2yHyEx4miOL8/5z9RIlGlL
HbGhBCpp99w0H8MPdBh6sdmbYtTm0k5aZcCh668NsufiUseOcgWtoRKAXe83qxhd
nR11utKj6yXkd/rWHo8enNjPbantTEiZS1aEj3tDz3NTnyUdt8C3LgHIZesoDOGM
aDL3+NW6wdjnHJGXJfzwD2KKD9lT1M07Qd+bcqKJHISBKabm3/StWI1b+lN/mPeM
T5DGpyzKB57UnfXpcqcpujus1IhTPGwjpJo/6A2wbiAbYsruOzrcE7BobDgZrp/E
P1UG+JlR0UeLqo0ySzIxccCWps7bUsavHB71Udbz4hOBQzBCUTMbREEj+bGdC3N6
YC5W6ZbHicyIbFOw5RwJ667gSETG0mSArUYehgHcmM7SrBzgr4i8EERBiGkayAvB
IHysRhYmQPULOMjz3tF0LHV7UBOduXnWJIWfWJkcXvJ5+fMOrm5aN/9LNpsuInkj
jL3pSMOPe+DfUcs4JCT3t9+DGuchWWu8iw4LFPlViS4jq+eLgiWzoznGnjHzQSM9
yI8NZZRYU9PuFWuWR9+EaQVVESW/WbCyUNSf3HsEEJFG7QgQNmOgDvD9pRKMxPz+
qaFFwMldFnYJOyJ8CCguX0Gl2S1N4r2VmyOCV0FSTc499VvJOUaX05bkrmHstbbx
ljDp2R0UYRwoamoDcTP1x91gf1UQB22+92vF1abKKW78sQTBrrO3rf9P0/OIDwNS
P+8GLBl4y+CRfPHDCrS0PTaiXLPIHHpnpTPWxCnt0XOXazdAxgG2oycva496zDCu
CHxYYnhYXtpvDWdCyv8WA5w3x9BvuW7RAyE9lnmerJHQQVp8rrOYCMajCXZcGv6o
Jbi7VhFlco6fJ8vqp+QoLEzrLw/WcB+C1215+ZGGriDsVzplITrfndc479Lvi9DJ
kcCLMYOFwHWzvodbxUweywM8RtX2sqmW8opVCMTx8rHnSltkLIHsry94b5QP5jSm
JY8O7FkqSDIgV7Xck8KucSuja28Ib0+AJZzePf334uzUaG0tFSamG+1cLD0CakHn
NIO/NRVJFGopWHDEODfKGDPvghl1eEB7E4l5IeQhHRCHUCqWXZGiGnDWFY/Hb+NU
IWPO1eGLeOcTE6sBpxXVrO5ouTEn4FhzgsK/uGrKM+mte++2/aTCr8a/9z0G1Ivs
qQWeoHWNMACW4LQxcDW0g7sWDgo6IxVmFL9fsCbUAB3tmZ4KMcEkmQMLO0jdGDRo
ZA+0BY/GsQqXSp/7qMlUHoiFjjElMnIa9h0UQvqRc6EyC5tdRBKp1BznCCPzVv2V
0IuSWzHRBCUodW4bZni5YoIfg4M/usfVaebJrSAZN+pJ1B990zNJpfZ+E2N+AiUN
HH2UDxFBtCTdjJvXsSKpW87iadQPynOGTFppHfwoQ/vFqOibtGvtC4fquvNHo7sh
W6MKEJnTCaVcFbvR7XnCdFh7+syNr0Qov7OaCWETp77DbQ337tDvZnvVe8q1ge0X
DgV1CaZKLQM7IqxfASaJdvsxGdSlYRKB5yKgsIRXJ6P6dcAFHLZUnHy/8OvHEedD
ZhPco1nCRiJAKYLVfeKAvr7DIsABTzavJspEL8o+kknri2juSc0xBGKEbRHHii9x
G5IdQNuIQyACkuXzr4ZYaPkd3z4J98pvkvnsJsNYC6umSVg/mjMqlNq9hX/XgpZI
G4L7JMkwYLyHUUc4ywY9Vh4ZKos9o4UAg7UE0Hy4kRBMoeyKJQts4RKt4rLOHsJ9
tIkxpgF1l7HzoyAlp0c72ZEjY6c2bLCpCLi2/5//wOM1w9Y9pJQOIe5VmPyYrfTZ
GLi2011mADp02FMPBhDzcFMxGKF9exCx8avRe8P2qe5agufis/OX9VdeFl3aSpmv
jR9Dr+BaXs614Y24oxqElwu5UL2WJrF0KG8svYcdsyLfwP/06PPBM/vAcGuy9ryj
4hX1hakWvqdVKqm6q+pw9Z6G1kwR3ehIWHzwniTS4ZW1OvmhxupOiKMSuSnLQEfA
krfBU7+DJ42alWquULOvgL6aIFAgiGNBwDoPN/fhkBxbGXhqL1a2siaMNT6UtIBh
34mATWKfjWHmIlOEs4v8cpQ11BnN/ciXMcU/dVHWtH3iA7R3VMjEIH/JHHhFIp/T
zjH06B5uiceL6rn+MV4wKe+zwy3Zd44ImWp3dk6Hr+nBYBRAGnAFInQPIwDMD3dh
H1XgGwxtNAomfgWbrCoZ0sR4ZC4OhwYuocDiZct1fCh9/O40Z63VJKPYVrioLNbw
6BF4Hh8lQZ42lzizGooJqb2OvpeGyOhZU8V1/AWnwn0cl9EZHzCWdXSuk2B4RQPM
27Q7xBJFrdZIFfvcuyuIGZnWjYOPJqBPXEm5ieO3gKIpEz2ofJmh7+UFsi13LqEv
6Ud/i0l6KMnqiwLwHcUDLlC9sbdF5Gjq7+Xlnpc5FI922/0WbMyJSV8mGoyr8Kh8
a7KwkvzGwGueQWJiocNIi4NQtlRbjaQyRCbdsB4Th0UlMccJAhUXIbGED5u1XWj3
P57+HAI2CFSESIlQV/IQMf06sYii4EllBoRpY3ltlch5X9laQ2rCOPmxdQcVuxgW
lOizZ12J8P+NwTe8CO+2Zun2D5msCoDA1s9GeTSknzd5pgO5z+EF4FvWcJEpXWoC
2qX8YgvkgmETJrUvUHu0pYNijj35I3e+PPwmd1mmd7ZWLw2LoSDdV8JYvGRyhL/H
G4Xw3mtrpsrtcEFklWx06F1w9YBSJdzSC4h703XUZQL5xjJRK5n/KEJ3mPdNLoQg
xLwmHP2UWJeBisfsziSWCnCnttVbabyd/1+lXd4IrIPY34Jm5chogiZFN0OlbODc
AFrX/b4sIgZBo9BK984btBWpo4DAw2Wn92C5awAQwuGtZ/M/bSIcpKYhF3N1IK9V
ytgzbdAdGlMGdpj//b+zrhMqzNIf+oDnGHbvvzJXOGPMP8lFgijA8U8QgNd4IzQi
h15R26kAs4bhaJINXj0p3n0Y7aqw4uc0aabhfGwpqSf/uMroIvsOdGxVtnk1l/Vg
835IKq+XAzaw/wGWy7S9voTp8f77TgRAvNfrf5cFJWoEL+mczd+HctXA6YCO/zFY
Rw0rLlwJcawiul43EXx32sW0IB295d/YbL3FwQNj7B/ZnL81jM/KZlC1FFN2D0tt
OQVfa//mOHDt5JThqW+w1FvoeOsg/NzVW3m2CVQQwSnN/O6gdE8ZRHXJb4uzzOmP
f09wf4JBsqECx9YSM3IkKVNINExPhNMS5A1P2pQQlGDPlpVTmlhd9XtqAg38MVKC
wBAsQq1mgjnHxSvextAF5ovMkM5TTD8ZHm17Loox1vmfoDfDL8uYjzrlsrA9iGOk
64XL0YQ2V5aJRoakiTmY5BM6bCP9FbdsuaVA5VbIu55AMdp6lxtYgGHW7dYqMJpf
f8uov2vUIk+FZskGFuu0kAz9ubczf8Cwl5sSsq8PxMsDGKbu34Kd9hWlVniAxpse
xrI6166w7OwRDvcO/mfLXEGXrnCY0Go8Hl11msJsU8+vkx6zi9yEtySoGVUgBadz
d/dVoMAMGorufJHRA4O2AKMXIMYkiaBh/Q+A2imEMLYWonpVOm2HAp9ue3+HC0R7
/oKDpiwYmcFbOWXTGbjISO+pleI8DkmnNOsitR3jLwjGt7mLx6Wew+H7SLPL2VLp
sp22j1iof68v4idfiMOiigRNYuFVZQrmDap0+S+2uhZV0ALBHTLu4lomcRRNrFEj
fwzjNu3PI9tl1cNXrXnCtEUe552gXU5uUKn16s1XZ03R55tk7pQTSeEAVqbpS9vM
b4dKfE5Srq4pxkC1ytaBCOC4lbe6XUsjftKL2Qxicu0ndU0yMb+gsnMHx1IIblZa
YkXGIktNkWWsVnVA+sBNkBzMcuejNCtCmpskTy1pFdTFQpgpBgaTDy8gi9UOIWrC
FIR7Fja1thXTpCS8FxPCNx4OqLP4XIlkQRcEnEgdOOeLGHjneQ9nepwz0f3/LMHh
XKFg3mnrv10+2lAX/0TiBVEjs+19R3ZxGZUnG1KRRSvNo2LQdcZ+yFpzPTpdNJM2
OZYPxBH2fECEVQmuEYEG1KxsStvxoTDEqETZZGDqm0K4o8sKfC6oksxUu6GTdtTb
8Wgw/tqfhlpEcPbqlKVG+KYNLHKYG9SVzJFn6FabWDrj3dYV6RvStTy1mrzxX3pR
DkabdnokDf1ZISC5I+Ah9A2jo7wiJoMdpQI9bSEK9kYR4YKcOK+1/pg7izxqwACH
Mo2J4QfZtspfMI0KXd/zBlRIX5gqy57wbUUOQgluVk87TAn5F5/u8ibhCEMrA64X
b9ex6BCDPioLMUsGVLSX3aOEXwyBClWjTJQzIZIP/di07TfZ7MMk6vH5BfJun/mq
TTt3XQ88pkM3fsRhSLXjfrhLRMvvGFCiToMQ32vx963HOadzMjIWjrYqyaVJrnqT
/eLVqr26I7vl/tzSKDugXgoPeXbJkL7jak7n6J1UzOfjVy6ksMoxmvSJgRAaAbOX
JycBN1K5jf0505RXy/cYX51atU9PzS1Im+oXrArdvl5Es1eqm1BpReAasOpujh6y
bAWg6azJ8CYYlTbOwiBxC84SWpp6rQ2Sgpb7Gh5bSMcXhMIj+7URNu6xRc4KFasm
9XgOzm7B+otL7GXTl5RzSNeLd9pCDR5CCcbyvu4AhphoJ3u2PJcRpCN1kLD7QQTp
F/7Oo0Qav2QkMxfhMeq4etBhHB25QAQBDS9aP2+dH7AoJRMREBvCJQLdPEGs21Ez
s2FwtlfytSWr8sOeNEv92X36ceuPAYLTrYRjps3b7G2CsIDSbnu458ysXQKhh34A
Hm+nF1K98YGO22peYYIvofdC5uBdB72n9gRcwepvhMiut2agsb7S1Nm5hMyHfcCS
hAtxGGHPFLJV7hV6HhgFDlrMVRCoqeNz48ZDFa5aVHZJQmGOuWADuFtP1U5j/VQ+
e+x4XHy4euMUIr6fDrSBSOVaWZvQuiqu2K9er7E8BUSVtf8qhFyPcxSUiyq+gcOe
rzc+QVOsXuss4UFc2YWzlhuUg2EL/o2r1zHFdwhzVRuTEpywVI5hHpIKtdq5UKLN
sE7g+YPKPJWUPTBaeNbFkOXOSIeVWd4Ch2eSO1EENbYVfJGmeJTR7vsmrupZHDw0
2Wuqd8jRp+y6PmYkBllnglC5DzbOmW8/8nBBRFHg7EdAswxNdpQDsYsyo4x91nxj
OLyp8GxObdnvRbNd36noZ1E9QJazVGUr7gkf0BMTV0lKZXa3dWKPJQgbtvqMGp6W
vKBC8m+/s9uyXfNFWY/GO73ys0ENH5pDsRa7vWWKwABsU8m7cXUt4/NqffodRuYb
pBoH2XJkh8WS5Dujmw91+9qHqoG2ujtBRy/swUJ/Q/5mGN3hAM7qZnmMzxGhufM9
41TIaKkvqlDNX/nN1nTPtkfwA6jczfyQiPniKcmpNHka7AgsBhLqb3rzIXkeDwvU
p0DTFDG3kdxcCrHzlY9hfnoeRaR4rY0wDJ9/ozH2CP5WXlGkUTErRoLwaHKmWCjv
97E25hJz+JIcMePgbkIhkHY/p9jNrrycG5nUoBrg2jlRiLDv5VfhSPPT5tO7jNvr
sNmh6wq5+4HsBWw2ovGog/uLBxPv/1wl/n+7K48MBJEuMFbHU7X6nrYs/hRXbI8I
xIgkOej2AJmt3gX1Cp62kwFHuxpjzay7omfijXXfutmKmbxzH2onljASnQCjj5YS
szFAhQDr4g+HNn3kPe/mJSU6yQy3DpwTn4u81W9rNv+wA2ZKg3+jWcLLr9AkKqnd
euLXCBznmukx9ejL7LL1wVPHoUPbT7fWEA9IYvEa6yXoM2NJcDTi04yKddx68ERO
3doBdizr2dfRbOpX4ADhuvkPzn/U9ss4C2ygHg0LtE7fbexeVLhXav6r31cwB2Fd
LeREtNIxqKWHO6EQlEXBjKu3e//mtkraCAcrjuh0GTXkSU8cpC/gH00WevMcScSV
6dYkKOzVwEcY6El6fFHEACCzdqagmH+BpeSN/r2ZdjBCQgLeKWumn7IsCzHDfRq6
Z58fTNCirOylGoqi1DnzcKgm108tl01VK8ZCYGbrzLKk6UGcdV2XJ+mPTkASFBzk
Tw8hbIfAcJwcsCakdIaVi+L57TjtlYAArxbpdk/uzsLWI2eTBiYVEaC2SmhsZWyP
Ef9PENTt0WejXPAGbzqmjBY+r4Ih/8etrWEf8Pj0tLw9zCwc47NYAtvI4eDzyevn
UtHfcp+aJ+oI8V1oE8PI5gDJouOc+5Zz7/uC4LcR96lzja9GYHFEKBEW8mTaVpEs
2c5lF922YBgc5P10L2JOMcAlDfOlX//iMKQKyldmREEOwpK15v5kXykYmSWdetQ8
Fwz3wFQ2UzbXOu95/jEw5QsTHdBKD++aOcnqXlkEjj3dyLD1B/olZbvNPoXJKKws
5C6qh6EjlIxPRAIjGCk2nuMMoKAi+8fnJn0y3YDA0f4JHoQ9vklLJWpT8CVZs10c
t8rfLIoM4qiWFgTDP8n3kDkG89UenTS3xOe1vttJDGy3Zkh+kMl3VxA29gwOi4ST
roxFL3+gmtE2V2FMpjD0RIzxmxKc6H3NTORSHoRGVfABSR8LLV0+Nf7YtO7ZkpHt
hCmn1UTPmfFISKZ3uWF14S2rqEge2wykZdjUKwurKLMXqW0fkwXfBSCLC+oXeifh
OSFNJJaM5xYW1ZzDgJURSQ8lEv8316JAPNFasssE+YxvHw2AKNau6CA6MrcHqUQy
vYNNDQJZEY90XvMVVkSNHEHaiNyCFz+kb6xvlVxFExlj75r9AtWx+YqxQh7tlvhM
MKFlNSIetK2VRRtGb6lCOQfvKzds6AUb3ZPLIBs0RKZ7Tk4HmQsPoCyaljP84Yhb
7BXOshy1DON8LhRrce79gVE97An61dqJVRlqQB3i9IH9n/j/0Wp5pYmdJcOsFrfz
NHt0NHH+wSS6DUE=
=0+in
-----END PGP MESSAGE-----

-----BEGIN PGP MESSAGE-----
Version: GnuPG v1.4.10 (Darwin)

hQIMA77xv53FwjH7ARAAp1IpXiQWYxkUHzdDiqJbTnEto0H295FykOHVkHVVTZt1
kIf9PNAvip/dZIQ6np9dSevaTzf5REtg/k18djWhk+BdyNoWF3efg0GpNZh/M6WU
Yd1PiGbzV/zSYee5+ifQhWMCXz4wxeo7ODIzge1dijLsOy8QIkWGmaIugJM5WzEB
3p092pYJ6/ZNP7ZvpgHQTHdTTIx9RrgZjSJxB+4X5+x2ItllM05NKl04dX+dlBIg
iRW534dm3K4wTWulPzKN4+zYaQ+ffQfQ6SUISUbGlYG1jDn/yZ2zFn6AyBHjjT4z
3bVMS+5BKU3BjJs+SCmu6psCSyuUaDo8wNFAlczdu2V8Lx2vjAPh4Q3JuZtEdcvm
mAa0FvaVK2UtavhdAT5UP6cwjdb8t88bQBwyWE9EgZVGYEP3ZWLOxHvxW3rp/zWN
dio8/e9Axxvf3Uvy3AxjV/hRg2jdLaEPkiT7QYRcZO3vsdnmsk2hQ1ZORfAmO7zK
+kfzYLirFxplzkZI29eq9H3tbPX9796XEkVXaeD4M5bzSNHtjfp3QpJ4uu5JnJxY
r6IPvb+Izzx1cEkNusltu1LO877+l115eNUNS17gRcWq6ApfJjO2r0HhUKWFMxT0
DGVtkxSGtqUn8/VQxuCeCvk54JdZYx/izS70eqBkcbdXsnC2P6U8s8KCI4gkaFPS
6gFcRdJLtXQ/LpERLTdulViZd844Z205BLWHPE+agTQfTyR3D7jH3NvnSGhF5hkj
A3wZVco5zUmeFKtkRdyVpdB9I8UuiI7qu7LLkqEwunQek/eNphuOSfDcIzs52pFs
g6W3MOp1yRuOTbsjYNXG/xQjPXuTJkJWJ2wQJcAi0yRku5YHn9Mkrl6On3wCUpLb
WWq0ynyzN1gm+oCsToTa4SiHXYywVtlrTOox3zim+eeVTkKaBKGtNwYj52ieI3jX
KSAc/J0gD+FeVP4luIfBr5UHVTRG7V8J7fvEpt0biUqwHduBi3fIsL6DX1BSGqln
xLQDqtiUgkpZZrgW2WAMJb3wdRASpapdQOjUHC0PDUO2LX3buJFfEFmGjA+ZCrG5
hEHY19VZAHbJ0mL0m4Js19ZkHxaL4IQUcVGtG1kDLV+JJlDXY1jPgh+qsYK3p/sA
mpA3bjug+/nnHAao4aMGUtT9V0aaoVv8innfO79plQ6nrVPwM71xr7bNUr3aWXbQ
nUgEX4wZFGS6aRyyJjE2bfs1Mlb2UWQ92cJbPV/SptnDTbPcw6u6H13LYeLpjCWk
VZ7Oi2HbifID9zc2o4r3xKeMn/BXIBRQFnnB8i6nePEkSJ09cvNAYJc/3UkEBRh6
Th11RYF3KBjgyeJ5dRbZCINpcdD2ZMwXw7ot9ggEQWznlWlKOMm3x96drudzisVn
tfPROkjOAL7/3tTDCytVnd8SxzzCbh+tr+h1ESH7V8jWTlAiCc5uN0NSPttyIW4F
8N2nm7wj3vMuyPQGwflSmPH4bN+z7912tL+iiYeGa9Uxiouqhs9arCMgkYJdYaQK
22A/9vYH2KjKenyMxdIPfUe4kcAv7hE1eginaFZDyu8W4p8AzUvyW4RT3sbda1zK
tX8EBxlh1UBy/6PRxBYCPVClPcvO4WgwZdwQjnhVevRk05DC3fzk7lJavx+larmY
ztpN+yMmGNQwN2W5a6MQUIYwkxbY6RfkKyYkAw/JQ+GxNpmHZ2xfNh0gUQBqov5e
n34yJysdKiOxTar57Bi1bcEFkk4k5gKfc7terIoqH5sbHlEAOi4OzrkXkIHsShrG
FN24i7AxTSC0s2RpjK3CW+VE3Zv04Wf2Ve9wPxmaOXdvxpNKcrZSAYtUrnTfdUPF
YyI24eh5Pw5ZyFbIWzFOuPTRt3Als3oSZcTbbuh7Vwb9vCvsXEoE6TLBEDNNz5+V
8X36QDWAHDgs6O/TfdmdWAZ6bkAoJnJO3Z9NIIXa5FNb9qox2mPeueu7/Gx7jq6I
D12SaIV36T8l77Bpt0bjOE7Xd+mzagqi72dxwO9Kr0fBahKIVlS/o7cgDTzMneuk
8OKBI6gQqzpwNjJC6LcqFGDpryzYSAvYc//xWRuye9G7yMR/Jf5fYu5hRUwKCuWy
Yup3taSxxp9Q7OO0W4SphTIhRdV+xZC9jarsregorAe2g6CdCEznODAjlGQQ+Vys
gGPNhrGXc6p588fSoo3ZVGnUGuCqErvh1Reyu05329SWEe1AAr3Bl4Cryu5wWemo
ZbwFhIkkt9NY17dSQ2xcKtRKlCfRxfjiCGIoLTBmWz7zCfs/2Tdieb6mofy3Ymqh
vniFMR2Pt7FXJOx78ttCsmN0dAqEB70teSaz25exTAIIyN0hvUt+SGUQJFbmELu9
fo8/U8c2MbIHkwFsBkPCbTIDF3rzEdHX6HM5lZVkDNIo6Tbeq3bVYRFG+kqZpX6G
gM7Y3SQjwHPumUuq4mVCS6wH/iDDYU5VZ2ObdV/3ohRGsuQa+QtcDpRADVszTv5U
q4ApXRwr8FsHsLUyaf+iWiyQKRocQo5bDb6Ec83l7G0gontKoInmJFP/Qvyrc1hO
aTuX2F1neGg+8skA2fE7LahdRnqo3j38aB8c2HwTQQPoehnV8SZLFUsnGCKlsMCh
Rs6YWfziLR8hp7xMXE47xXAlZWHReMpcxdx6M9HJ0O4rj+X2dSKLqEq/p8+3RF0F
mgLVZ+WkuuOVXy+xvSwvDFdqV+t2TIRhfNblK2CiGgyUHFmr76FBtZmzXqd8uUiu
u3DADoCil9zZfqHMaivHgqVxHXJNEZW+G4Usaqk7zLbAsFkVBJFK3eid9eMTduEp
g/R2DsHHm6mmxOG/OqiMj/hqHFpuDumgJYF8EeFVQjpRVKEN0A06k/W0MifA1wq1
6gmOGdxNyCnSVRh2BXLJoz0wMqXWN8Dcf87fhS7W7qRo+qmzsa65mFXpeaoqiTMg
UcGunNVEEvxswkdpidSIRwuDNSXSivVSe9tZTC3iAEusYJ96IbpNHkjSvVtY7H08
OE6sOgMYs5pfcq2mot1F07B+
=a7uA
-----END PGP MESSAGE-----

\section{Conlusion} % (fold)
\label{sec:conlusion}

In this paper, we used $\pi-calculus$ and P/T nets to specify and analyze a social networking application, {\tt Social News}. Our specifications showed two levels of interactions: within a group with
secrecy and confidentiality and at a global level. As for the analysis, we focused on verifying the correctness of the interactions and the achievability of the overall goal of the application.

In introduction to this paper, we claimed that one should turn to algebraic approaches instead of the purely statistical ones, in order to gain a deeper understanding of behaviors in social networks.
However, our claim does not suggest to squarely rule out the statistical approaches. Rather, we advocate for a hybrid approach, given the wealth of data/information available in social networks.
Indeed, the algebraic approaches could provide a concise and accurate model of the social networking application at hand, while the statistical approaches help identify new trends or development of the
initial model for further refinement. In short, our future research will seek to reconcile these two types of approaches.

By attempting to analyze a social networking using algebraic approaches, this research aims at addressing more complex questions such as \emph{how does the behavior of a social network evolve over
time?} Such understanding can help further automate processes within a social network and reach the ideal situation where humans interact with computers in a transparent and flexible manner.

In this paper, we have adopted $CTL^*$ as a formalism to represent a task being carried out. Then we represented the execution of the task using P/T net on top of $\pi-calculus$ terms. In the future,
we wish to investigate how to model check the execution of the task using the P/T net. More generally, following the foundation laid by Groote and Reniers (see~\cite{Groote-Reniers:01}), we wish to
investigate more appropriate methodologies for process algebraic verification in social networking applications.

Another important question raised in this research is \emph{task division}. Given a task submitted to a community in a social network, how to come up with a possibly automated fair division of the task
into subtasks among its members.

% section conlusion (end)

\bibliographystyle{acmtrans}
\bibliography{socnet}

\begin{received}
	Received unknown;
	accepted unknown;
\end{received}

\end{document}

% Local Variables:
% ispell-local-dictionary: "american"
% End: