\section{Global Headline Adoption} % (fold)
\label{sec:global_headline_adoption}

\textbf{Need to bring this section closer to the others}

At this point, {\tt Social News} has already filtered local headlines from the initial submissions. These headlines will
now compete at a global level to adopt the final ones. In this section we model the global headline adoption step.

Note that before beginning this step, the two properties mentioned in section~\ref{sub:formalization}: unicity of news
submission and non inclusion of groups should be checked. However, this discussion is out of the scope of this paper,
since we do not tackle model checking in this paper.

In order to provide the reader with a better insight of what is concurrently taking place in the various groups we change
our specification formalism from $\pi-calculus$ to \emph{P/T nets}. While $\pi-calculus$ proved to be very efficient at
the local level, it becomes cumbersome at a global level. Besides, some of the requirements at the global level are more
difficult to capture in $\pi-calculus$. For example, a group should not vote for its own news. In order to express such a
property in $\pi-calculus$ we need to introduce data annotation on top of $\pi-calculus$. Finally, previous research has
established a close connection between both formalisms. Indeed, Petri nets are often used to provide a semantics of
$\pi-calculus$ expressions (e.g., see~\cite{Devillers-Klaudel-Koutny:06}). In pursuit of simplicity, we have resorted to
use a \emph{P/T net} to model the global headline adoption.

At the beginning of this step, each group has its set of local headlines generated from the previous step. A group can
then publish its headlines to the \emph{wall} or try to form a coalition. Generally, a coalition is a group formed in
order to increase the individual or overall utility function of its members. In our case, it increases the members'
chances to pass news as global headlines. A coalition, when formed, should combine all the small groups which took part
into a bigger one. When the coalition is successful, all the groups which joined hand can now published their new set of
headlines. When the coalition fails, each group returns back to its local headlines. Then, the group can either publish
its headlines or try another coalition.

Once a group has published its headline, other groups can vote for them. In order to vote, a group must first execute
silent transitions ($\tau$) and then select news published by another group and vote for them. All the votes are
collected in a single place, \emph{global HL}. Figure~\ref{fig:sn-news-Petri} depicts our P/T representation of the
global headline adoption involving three groups: $g^{\nu_\imath}$, $g^{\nu_\jmath}$ and $g^{\nu_k}$. The figure shows
both the case where groups vote individually and form a coalition ($g^{\nu_\imath}$ and $g^{\nu_k}$).

\begin{figure}
	\centering 
	\includegraphics[width=.85\textwidth]{socnet-Petri} 
	\caption{Global Headline Adoption}
	\label{fig:sn-news-Petri} 
\end{figure}

With the P/T net depicted in figure~\ref{fig:sn-news-Petri}, we can now check whether the goal defined in
equation~\ref{eq:appgoal} is achievable. In order to do so, we simply have to check if from the initial marking $M_0$,
where each group has its local headlines, we can reach a marking $M$ where exactly $N$\footnote{Note that because of
silent transitions for voting we should deduct as many tokens which transited by the \emph{vote} places from the tokens
in \emph{global HL} to obtain $N$.} news are in \emph{global HL}. The advantage in checking that property, is for example
to quickly alert the different groups when the local headlines are not enough to reach the overall goal.

% section global_headline_adoption (end)